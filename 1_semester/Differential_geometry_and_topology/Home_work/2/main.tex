\documentclass{article}

\usepackage[T2A]{fontenc}
\usepackage[utf8]{inputenc} 
\usepackage[english,russian]{babel} 
\usepackage{graphicx} 
\usepackage{amsmath}
\usepackage{amssymb}
\usepackage{cancel}
\usepackage{amsfonts}
\usepackage{titlesec}
\usepackage{titling} 
\usepackage{geometry}
\usepackage{pgfplots}
\usepackage{esint}
\pgfplotsset{compat=1.9}
\usepackage{amsthm}


\titleformat{\section}
{\normalfont\Large\bfseries}{\arabic{section}}{1em}{}
\titleformat{\subsection}
{\normalfont\large\bfseries}{}{1em}{}


\setlength{\droptitle}{-3em} 
\title{\vspace{-1cm}Домашняя работа №2 \\ по дисциплине "Дифференциальная геометрия и топология"}
\author{Винницкая Дина Сергеевна}
\date{Группа: Б9122-02.03.01сцт}

\geometry{a4paper, margin=2cm}

\begin{document}
	
	\maketitle

        \section{Задание} 

       \begin{enumerate}
            \item Доказать, что открытые множества, введённые в доказательстве бесконечности $\mathbb{P}$, образуют топологию на $\mathbb{Z}$.
            
            \item Набор $\Sigma$ открытых множеств является базой  $\tau$:
            \[
            \tau \iff \forall U \in E \ \forall x \in U \ \exists \mathbb{Z}  \subset U : x \in \mathbb{Z}  \subset U
            \]
        \end{enumerate}


        \section{Доказать, что открытые множества, введённые в доказательстве бесконечности $\mathbb{P}$, образуют топологию на $\mathbb{Z}$.} 
        
        \subsection{Решение:}

        Известно, что для того чтобы множество являлось топологией, оно должно удовлетворять следующим условиям:
        
        \[
        \emptyset, \mathbb{Z}  \in \tau
        \]
        
        По определению, $U$ может быть пустым множеством $\emptyset$.
        
        Также $\forall a \in U \ \exists b = 1 > 0 \in \mathbb{Z} : U = \mathbb{Z}  \subset \mathbb{Z}  \implies \mathbb{Z}  \in \tau$
        
        \textbf{Если} $\{U_i \in \tau | i \in J\}$, \textbf{то} объединение $\left( U_1 \cup U_2 \cup \dots \cup U_n \right) \in \tau$
        
        \[
        U = \bigcup U_i \implies \forall a \in U \ \exists  U_i : a \in U_i
        \]
        \[
        U_i \in \tau \implies \exists b > 0 \in \mathbb{Z}  : N_{a,b} \subset U_i \subset U \implies N_{a,b} \subset U
        \]
        
        Следовательно:
        \[
        \forall a \in U \ \exists b > 0 \in \mathbb{Z}  : N_{a,b} \subset U \implies U \in \tau
        \]
        
        \[
        \forall U_1, U_2 \in \tau; \quad U_1 \cap U_2 \in \tau
        \]
        
        \[
        U = \bigcap U_i
        \]
        \[
        \forall a = a_1 = a_2 \in U \ \exists b = \prod_{i=1}^{2} b_i > 0 \in \mathbb{Z}  : N_{a,b} \subset U_i \implies N_{a,b} \subset U \implies U \in \tau
        \]
        \[
        \{a + b \cdot k \cdot n \} \subset \{a + b \cdot n \}
        \]
        \[
        \qed 
        \]

        \section{Набор $\Sigma$ открытых множеств является базой  $\tau$:
            \[
            \tau \iff \forall U \in \tau \ \forall x \in U \ \exists \mathbb{V} \in \Sigma : x \in \mathbb{V} \subseteq  U
            \]} 
        
        \subsection{Решение:}
        Из определения базы топологии $B$. Пусть набор $B = \{V_i | i \in I \}$:
        \[
        B = \{V_i\} : \forall U \in \tau \implies \exists I \subset \mathbb{N} : U = \bigcup_{i \in I} V_i
        \]
        
        \textbf{Необходимое условие}:
        Необходимо доказать, что если $\Sigma$ — база, то выполняется данное условие.
        
        Предпологаем, что $\Sigma$ — база топологии:
        \[
        \forall U \in \tau \implies U = \bigcup_{i \in I} V_i, \quad V_i \in \Sigma
        \]
        \[
        U = \bigcup V_i \implies \forall x \in U \ \exists V_i : x \in V_i
        \]
        \[
        \implies \forall x \in U \ \exists V_i \in \Sigma : x \in V_i  \subset U
        \]
        
        \textbf{Достаточное условие}:
        Теперь покажем, что если выполняется данное условие, то $\Sigma$ является базой.
        
        Из условия следует, что
        \[
        \exists I \subset \mathbb{N} : V_i \subset U, \quad i \in I
        \]
        \[
        \forall U \in \tau \ \exists I \subset \mathbb{N} : U = \bigcup_{i \in I} V_i
        \]
        Следовательно, $\Sigma = \{V_i\}$ — это база топологии.
        
        \[
        \qed 
        \]
               
        
\end{document}
