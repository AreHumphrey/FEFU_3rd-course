\documentclass{article}

\usepackage[T2A]{fontenc}
\usepackage[utf8]{inputenc} 
\usepackage[english,russian]{babel} 
\usepackage{graphicx} 
\usepackage{amsmath}
\usepackage{amssymb}
\usepackage{cancel}
\usepackage{amsfonts}
\usepackage{titlesec}
\usepackage{titling} 
\usepackage{geometry}
\usepackage{pgfplots}
\usepackage{esint}
\pgfplotsset{compat=1.9}
\usepackage{amsthm}


\titleformat{\section}
{\normalfont\Large\bfseries}{\arabic{section}}{1em}{}
\titleformat{\subsection}
{\normalfont\large\bfseries}{}{1em}{}


\setlength{\droptitle}{-3em} 
\title{\vspace{-1cm}Домашняя работа №9 \\ по дисциплине "Дифференциальная геометрия и топология"}
\author{Винницкая Дина Сергеевна}
\date{Группа: Б9122-02.03.01сцт}

\geometry{a4paper, margin=2cm}

\begin{document}
	
	\maketitle


        \subsection*{Условие задачи}
        Найти кривизну и кручение:
        \begin{enumerate}
                \item Окружности
                \item Винтовой линии
                \item $\vec{\gamma}(t) = \left(t - \sin(t), 1 - \cos(t), 4 \sin\left(\frac{t}{2}\right)\right), \quad t = \pi$
        \end{enumerate}
            
        \section*{Решение}
       \subsubsection*{Выкладки}
        Кривизна:
        \[
        k(t) = \frac{\|\vec{\gamma}' \times \vec{\gamma}''\|}{\|\vec{\gamma}'\|^3}.
        \]
        
        Кручение:
        \[
        \Xi(t) = \frac{(\vec{\gamma}', \vec{\gamma}'', \vec{\gamma}''')}{\|\vec{\gamma}' \times \vec{\gamma}''\|^2}.
        \]
        
        \subsubsection*{1. Окружность}
        
        Задана окружность:
        \[
        \vec{\gamma}(t) = (r \cos(t), r \sin(t), c).
        \]
        
        \textbf{Производные:}
        \[
        \vec{\gamma}'(t) = (-r \sin(t), r \cos(t), 0),
        \]
        \[
        \vec{\gamma}''(t) = (-r \cos(t), -r \sin(t), 0),
        \]
        \[
        \vec{\gamma}' \times \vec{\gamma}'' = (0, 0, r^2).
        \]
        
        \textbf{Модули:}
        \[
        \|\vec{\gamma}'\| = r, \quad \|\vec{\gamma}' \times \vec{\gamma}''\| = r^2.
        \]
        
        \textbf{Кривизна:}
        Кривизна вычисляется по стандартной формуле. Для окружности:
        \[
        k(t) = \frac{\|\vec{\gamma}' \times \vec{\gamma}''\|}{\|\vec{\gamma}'\|^3} = \frac{r^2}{r^3} = \frac{1}{r}.
        \]
        Это означает, что кривизна окружности обратна её радиусу $r$.
        
        \textbf{Кручение:}
        Так как окружность — двумерная фигура, кручение определяется, но его значение всегда равно нулю:
        \[
        \Xi(t) = 0.
        \]
        Физически это связано с тем, что окружность не имеет третьей координаты для "закручивания" в пространстве.
        
        Таким образом:
        \begin{itemize}
            \item Кривизна окружности равна $\frac{1}{r}$.
            \item Кручение существует и равно нулю.
        \end{itemize}
        
        \subsubsection*{2. Винтовая линия}
        
        Задана винтовая линия:
        \[
        \vec{\gamma}(t) = \left(r \cos(t), r \sin(t), ht\right).
        \]
        
        \textbf{Производные:}
        \[
        \vec{\gamma}' = \left(-r \sin(t), r \cos(t), h\right),
        \]
        \[
        \vec{\gamma}'' = \left(-r \cos(t), -r \sin(t), 0\right),
        \]
        \[
        \vec{\gamma}''' = \left(r \sin(t), -r \cos(t), 0\right).
        \]
        
        \textbf{Векторное произведение:}
        \[
       [ \vec{\gamma}' ,  \vec{\gamma}'' ] = 
        \begin{pmatrix}
        hr \sin(t) \\
        -hr \cos(t) \\
        r^2
        \end{pmatrix}.
        \]
        
        \textbf{Модули:}
        \[
        \|\vec{\gamma}' \| = \sqrt{r^2 + h^2}, \quad
        \|\vec{\gamma}',  \vec{\gamma}''\| = r\sqrt{h^2 + r^2}.
        \]
        
        \textbf{Кривизна:}
        Кривизна вычисляется как:
        \[
        k(t) = \frac{\|\vec{\gamma}' \times \vec{\gamma}''\|}{\|\vec{\gamma}'\|^3} = \frac{r}{\left(h^2 + r^2\right)}.
        \]
        
        \textbf{Смешанное произведение:}
        \[
        (\vec{\gamma}', \vec{\gamma}'', \vec{\gamma}''') = hr^2.
        \]
        
        \textbf{Кручение:}
        Кручение вычисляется как:
        \[
        \Xi = \frac{(\vec{\gamma}', \vec{\gamma}'', \vec{\gamma}''')}{\|\vec{\gamma}' \times \vec{\gamma}''\|^2} = \frac{h}{h^2 + r^2}.
        \]
        
        \textbf{Вывод:}
        \begin{itemize}
            \item Кривизна винтовой линии равна:
            \[
            k(t) = \frac{r}{h^2 + r^2}.
            \]
            \item Кручение винтовой линии существует и вычисляется как:
            \[
            \Xi(t) = \frac{h}{h^2 + r^2}.
            \]
        \end{itemize}
    \subsubsection*{3. Пространственная кривая}
        
        Задана пространственная кривая:
        \[
        \vec{\gamma}(t) = \left(t - \sin(t), 1 - \cos(t), 4 \sin\left(\frac{t}{2}\right)\right), \quad t = \pi.
        \]
        
        \textbf{Производные:}
        \[
        \vec{\gamma}'(t) = \left(1 - \cos(t), \sin(t), 2 \cos\left(\frac{t}{2}\right)\right),
        \]
        \[
        \vec{\gamma}''(t) = \left(\sin(t), \cos(t), -\sin\left(\frac{t}{2}\right)\right),
        \]
        \[
        \vec{\gamma}'''(t) = \left(\cos(t), -\sin(t), -\frac{1}{2} \cos\left(\frac{t}{2}\right)\right).
        \]
        
        \textbf{Вычисление в точке $t = \pi$:}
        \[
        \vec{\gamma}(\pi) = \left(\pi, 1 + 1, 0\right) = \left(\pi, 2, 0\right),
        \]
        \[
        \vec{\gamma}'(\pi) = \left(2, 0, 0\right),
        \]
        \[
        \vec{\gamma}''(\pi) = \left(0, 1, -1\right),
        \]
        \[
        \vec{\gamma}'''(\pi) = \left(1, 0, 0\right).
        \]
        
        \textbf{Векторное произведение:}
        \[
        \vec{\gamma}' , \vec{\gamma}'' = \left(0, 2, 2\right).
        \]
        
        \textbf{Модули:}
        \[
        \|\vec{\gamma}' , \vec{\gamma}''\| = 2\sqrt{2}, \quad \|\vec{\gamma}'\| = 2.
        \]
        
        \textbf{Смешанное произведение:}
        \[
        (\vec{\gamma}', \vec{\gamma}'', \vec{\gamma}''') = 0.
        \]
        
        \textbf{Кривизна:}
        Кривизна рассчитывается по формуле:
        \[
        k(t) = \frac{\|\vec{\gamma}' , \vec{\gamma}''\|}{\|\vec{\gamma}'\|^3} = \frac{2\sqrt{2}}{2^3} = 2^{-\frac{3}{2}}.
        \]
        
        \textbf{Кручение:}
        Кручение рассчитывается по формуле:
        \[
        \Xi = \frac{(\vec{\gamma}', \vec{\gamma}'', \vec{\gamma}''')}{\|\vec{\gamma}' , \vec{\gamma}''\|^2} = 0.
        \]
        \subsection*{Условие задачи}
        Доказать, что кривая плоская:
        \[
        \vec{\gamma}(t) = \left(2t, 3t + 4, t^2 - 3\right).
        \]
        
        Найти уравнение плоскости, в которой лежит эта кривая.
            
        \section*{Решение}
        \[\textbf{Способ через нормаль к соприкасающейся |bad|}\\
        
        Найдем базис Френе для данной кривой:
        \[
        \vec{\gamma}(t) = \left(2t, 3t + 4, t^2 - 3\right).
        \]
        Первая производная:
        \[
        \vec{\gamma}'(t) = \left(2, 3, 2t\right).
        \]
        Нормируем вектор производной:
        \[
        \vec{v}(t) = \frac{\vec{\gamma}'(t)}{\|\vec{\gamma}'(t)\|} = \frac{\left(2, 3, 2t\right)}{\sqrt{4 + 9 + 4t^2}} = \left(\frac{2}{\sqrt{4t^2 + 13}}, \frac{3}{\sqrt{4t^2 + 13}}, \frac{2t}{\sqrt{4t^2 + 13}}\right).
        \]
        Вторая производная:
        \[
        \vec{v}'(t) = \left(-\frac{8t}{(4t^2 + 13)^{3/2}}, -\frac{12t}{(4t^2 + 13)^{3/2}}, \frac{4}{\sqrt{4t^2 + 13}} - \frac{4t^2}{(4t^2 + 13)^{3/2}}\right).
        \]
        Модуль вектора второй производной:
        \[
        \|\vec{v}'\| = \frac{26}{4t^2 \sqrt{13} + 13^{3/2}}.
        \]
        Комментарий: данный способ требует нахождения второго и третьего производных и анализа нормали. Однако он является сложным и неоптимальным для задачи проверки плоскостности, поскольку метод базиса Френе избыточен.
        \[\textbf{Способ через кручение}\]

        Используем кривую:
        \[
        \vec{\gamma}(t) = \left(2t, 3t + 4, t^2 - 3\right).
        \]
        Первая производная:
        \[
        \vec{\gamma}'(t) = \left(2, 3, 2t\right).
        \]
        Вторая производная:
        \[
        \vec{\gamma}''(t) = \left(0, 0, 2\right).
        \]
        Третья производная:
        \[
        \vec{\gamma}'''(t) = \left(0, 0, 0\right).
        \]
        Смешанное произведение:
        \[
        (\vec{\gamma}', \vec{\gamma}'', \vec{\gamma}''') = 0.
        \]
        Кручение вычисляется как:
        \[
        \Xi = \frac{(\vec{\gamma}', \vec{\gamma}'', \vec{\gamma}''')}{\|\vec{\gamma}' \times \vec{\gamma}''\|^2}.
        \]
        Поскольку числитель равен нулю, кручение:
        \[
        \Xi = 0.
        \]
        Вывод: кручение равно нулю, следовательно, кривая плоская.
        \subsubsection*{Нахождение соприкасающейся плоскости}
        Построим плоскость по 3 точкам при $t \in \{-1, 0, 1\}$:
        \[
        \vec{\gamma}(-1) = (-2, 1, -2), \quad \vec{\gamma}(0) = (0, 4, -3), \quad \vec{\gamma}(1) = (2, 7, -2).
        \]
        
        Уравнение плоскости запишем через определитель:
        \[
        \begin{vmatrix}
        x - x_A & y - y_A & z - z_A \\
        x_B - x_A & y_B - y_A & z_B - z_A \\
        x_C - x_A & y_C - y_A & z_C - z_A
        \end{vmatrix} = 0.
        \]
        
        Подставляем координаты точек:
        \[
        \begin{vmatrix}
        x + 2 & y - 1 & z + 2 \\
        2 & 3 & -1 \\
        4 & 6 & 0
        \end{vmatrix} = 0.
        \]
        
        Раскрываем определитель:
        \[
        (x + 2)(3 \cdot 0 - 6 \cdot (-1)) - (y - 1)(2 \cdot 0 - 4 \cdot (-1)) + (z + 2)(2 \cdot 6 - 4 \cdot 3) = 0.
        \]
        
        Упрощаем:
        \[
        (x + 2)(6) - (y - 1)(-4) + (z + 2)(0) = 0.
        \]
        
        Далее:
        \[
        6x + 12 + 4y - 4 = 0.
        \]
        
        Итоговое уравнение плоскости:
        \[
        6x + 4y + 8 = 0.
        \]
        
        Ответ:
        \[
        3x - 2y + 8 = 0.
        \]

        
                        
               
                                      
                
\end{document}
