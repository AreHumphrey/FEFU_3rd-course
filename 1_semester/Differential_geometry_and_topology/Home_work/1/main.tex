\documentclass{article}

\usepackage[T2A]{fontenc}
\usepackage[utf8]{inputenc} 
\usepackage[english,russian]{babel} 
\usepackage{graphicx} 
\usepackage{amsmath}
\usepackage{amssymb}
\usepackage{cancel}
\usepackage{amsfonts}
\usepackage{titlesec}
\usepackage{titling} 
\usepackage{geometry}
\usepackage{pgfplots}
\usepackage{esint}
\pgfplotsset{compat=1.9}


\titleformat{\section}
{\normalfont\Large\bfseries}{\arabic{section}}{1em}{}
\titleformat{\subsection}
{\normalfont\large\bfseries}{}{1em}{}


\setlength{\droptitle}{-3em} 
\title{\vspace{-1cm}Домашняя работа №1 \\ по дисциплине "Дифференциальная геометрия и топология"}
\author{Винницкая Дина Сергеевна}
\date{Группа: Б9122-02.03.01сцт}

\geometry{a4paper, margin=2cm}

\begin{document}
	
	\maketitle

	\section{Доказать являются ли вектора базисами} 
        
        \[
        a_1 = \begin{pmatrix} 1 \\ 1 \\ 0 \end{pmatrix}, \quad a_2 = \begin{pmatrix} 1 \\ 0 \\ 1 \end{pmatrix}, \quad a_3 = \begin{pmatrix} 0 \\ 1 \\ 1 \end{pmatrix}
        \]
        \[
        b_1 = \begin{pmatrix} 1 \\ 2 \\ 2 \end{pmatrix}, \quad b_2 = \begin{pmatrix} 2 \\ 1 \\ 2 \end{pmatrix}, \quad b_3 = \begin{pmatrix} 2 \\ -2 \\ 1 \end{pmatrix}
        \]
        
	\subsection{Решение:}

        \[
        A = \begin{pmatrix}
        1 & 1 & 0 \\
        1 & 0 & 1 \\
        0 & 1 & 1
        \end{pmatrix}
        = -1 - 1 = -2 \neq 0  \Longrightarrow  \text{линейно независимые вектора} \Longrightarrow  \text{базис}
        \]
        \[
        B = \begin{pmatrix}
        1 & 2 & 2 \\
        2 & 1 & -2 \\
        2 & 2 & 1
        \end{pmatrix}
         = 1 - 2 \cdot 2 = 1 - 4 = -3 \neq 0  \Longrightarrow \text{линейно независимые вектора}  \Longrightarrow  \text{базис}
        \]
        \subsection{Ответ: $\textbf{Являются базисами}$}

        \section{Найти матрицу перехода от базиса $\{a\}$ к базису $\{b\}$} 
        \subsection{Решение:}

        \[
        \begin{cases}
         b_1 = \alpha^{1}_1 a_1 + \alpha^{2}_1  a_2 + \alpha^{3}_1  a_3 \\
        b_2 = \alpha^{1}_2  a_1 + \alpha^{2}_2 a_2 + \alpha^{3}_2  a_3 \\
        b_3 = \alpha^{1}_3  a_1 + \alpha^{2}_3 a_2 + \alpha^{3}_3  a_3
        \end{cases}
        \Longrightarrow
        \]
        \[
        \Longrightarrow
        \begin{cases}
        b_1 = \frac{1}{2} a_1 + \frac{1}{2} a_2 + \frac{3}{2} a_3 \\
        b_2 = \frac{1}{2} a_1 + \frac{3}{2} a_2 + \frac{1}{2} a_3 \\
        b_3 = -\frac{1}{2} a_1 + \frac{5}{2} a_2 - \frac{3}{2} a_3
        \end{cases}
        \]
        
        Из этого следует то, что матрицей перехода будет матрица вида: \( A^T = \frac{1}{2} \begin{pmatrix}
        1 & 1 & 3 \\
        1 & 3 & 1 \\
        -1 & 5 & -3
        \end{pmatrix} \)

        \subsection{Ответ: $ A^T = \frac{1}{2} \begin{pmatrix}
        1 & 1 & 3 \\
        1 & 3 & 1 \\
        -1 & 5 & -3
        \end{pmatrix}$}
        
        \section{Найти координаты вектора $X = \begin{pmatrix} -1 \\ 2 \\ 7 \end{pmatrix}$ в базисах $\{a\}$ и $\{b\}$} 
        
        \subsection{Решение:}
        
        
        \[
        \left\{
        \begin{aligned}
        X_a &= \alpha_1 a_1 + \alpha_2 a_2 + \alpha_3 a_3 \\
        X_b &= \beta_1 b_1 + \beta_2 b_2 + \beta_3 b_3
        \end{aligned}
        \right.
        \]

        \[
        \Longrightarrow
        \left\{
        \begin{aligned}
        X_a &= -\frac{5}{2} a_1 + \frac{1}{2} a_2 + \frac{3}{2} a_3 \\
        X_b &= \frac{29}{3} b_1 - 9 b_2 + \frac{17}{3} b_3
        \end{aligned}
        \right.
        \]
        Таким образом получаем следующие координаты:
        \[
        \textbf{Координаты $X$ в базисе $a$} \quad \Longrightarrow \quad
        \left(
        \begin{array}{c}
        -\frac{5}{2} \\
        -1 \\
        \frac{3}{2}
        \end{array}
        \right)
        \]

        \[
        \textbf{Координаты $X$ в базисе $b$} \quad\Longrightarrow \quad
        \left(
        \begin{array}{c}
        \frac{29}{3} \\
        -9 \\
        \frac{17}{3}
        \end{array}
        \right)
        \]

        \section{Найти координаты $y = 3a_1 - a_2 + 7a_3$ в базисе $\{b\}$} 
        
        \subsection{Решение:}
        Запишем  уравнение $y = 3a_1 - a_2 + 7a_3$ в виде 
                \[
        Y_a = \left(
        \begin{array}{c}
        3 \\
        -1 \\
        7
        \end{array}
        \right)
        \]

        Подставим значения \( a_1, a_2, a_3 \):

        \[
        Y_a = 3 \begin{pmatrix} 1 \\ 1 \\ 0 \end{pmatrix} - \begin{pmatrix} 1 \\ 0 \\ 1 \end{pmatrix} + 7 \begin{pmatrix} 0 \\ 1 \\ 1 \end{pmatrix} = \begin{pmatrix} 3 - 1 + 0 \\ 3 - 0 + 7 \\ 0 - 1 + 7 \end{pmatrix} = \begin{pmatrix} 2 \\ 10 \\ 6 \end{pmatrix}
        \]
        Поскольку, 
        
        \[
        b_1 = \begin{pmatrix} 1 \\ 2 \\ 2 \end{pmatrix}, \quad b_2 = \begin{pmatrix} 2 \\ 1 \\ 2 \end{pmatrix}, \quad b_3 = \begin{pmatrix} 2 \\ -2 \\ 1 \end{pmatrix} \Longrightarrow 
        B = \begin{pmatrix}
        1 & 2 & 2 \\
        2 & 1 & -2 \\
        2 & 2 & 1
        \end{pmatrix}
        \]
        Таким образом, получаем:
        
        \[
        \begin{pmatrix} 2 \\ 10 \\ 6 \end{pmatrix} = c_1 \begin{pmatrix} 1 \\ 2 \\ 2 \end{pmatrix} + c_2 \begin{pmatrix} 2 \\ 1 \\ 2 \end{pmatrix} + c_3 \begin{pmatrix} 2 \\ -2 \\ 1 \end{pmatrix}
        \]
                
        
        Таким образом, координаты уровнения в базисе  $\{b\}$:
        \[
        Y_b = \begin{pmatrix} 2 \\ 2 \\ -2 \end{pmatrix} \Longrightarrow y = 2b_1 + 2b_2 - 2b_3
        \]
        
        \subsection{Ответ: $ \begin{pmatrix} 2 \\ 2 \\ -2 \end{pmatrix}$}
        
        
\end{document}
