\documentclass{article}

\usepackage[T2A]{fontenc}
\usepackage[utf8]{inputenc} 
\usepackage[english,russian]{babel} 
\usepackage{graphicx} 
\usepackage{amsmath}
\usepackage{amssymb}
\usepackage{cancel}
\usepackage{amsfonts}
\usepackage{titlesec}
\usepackage{titling} 
\usepackage{geometry}
\usepackage{pgfplots}
\usepackage{esint}
\pgfplotsset{compat=1.9}
\usepackage{amsthm}


\titleformat{\section}
{\normalfont\Large\bfseries}{\arabic{section}}{1em}{}
\titleformat{\subsection}
{\normalfont\large\bfseries}{}{1em}{}


\setlength{\droptitle}{-3em} 
\title{\vspace{-1cm}Домашняя работа №11 \\ по дисциплине "Дифференциальная геометрия и топология"}
\author{Винницкая Дина Сергеевна}
\date{Группа: Б9122-02.03.01сцт}

\geometry{a4paper, margin=2cm}

\begin{document}
	
	\maketitle


        \subsection*{Условие задачи}
         Вывести уравнение соприкасающейся окружности для произвольной параметризации.
        
        \section*{Решение}
        Рассмотрим кривую, заданную параметрически:
        \[
        \gamma(t) = (x(t), y(t)),
        \]
        а также окружность с радиусом $R$, заданную в параметрическом виде:
        \[
        r(t) = (R \cos t, R \sin t).
        \]
        Чтобы найти уравнение соприкасающейся окружности, нужно определить её кривизну, радиус и вектор нормали.
        
        \subsection*{№1 Нахождение кривизны окружности}
        Кривизна определяется как мера изгиба кривой в точке. Для параметрической кривой её значение вычисляется по формуле:
        \[
        k(t) = \frac{\| r'(t), r''(t) \|}{|r'(t)|^3}.
        \]
        Сначала найдем первую производную $r'(t)$ по параметру $t$:
        \[
        r(t) = (R \cos t, R \sin t),
        \]
        \[
        r'(t) = \left(\frac{d}{dt}(R \cos t), \frac{d}{dt}(R \sin t)\right) = (-R \sin t, R \cos t).
        \]
        Теперь найдём вторую производную $r''(t)$:
        \[
        r''(t) = \left(\frac{d}{dt}(-R \sin t), \frac{d}{dt}(R \cos t)\right) = (-R \cos t, -R \sin t).
        \]
        Далее вычислим длины производных:
        \[
        |r'(t)| = \sqrt{(-R \sin t)^2 + (R \cos t)^2} = \sqrt{R^2 (\sin^2 t + \cos^2 t)} = R,
        \]
        \[
        |[ r'(t), r''(t) |] = |R|^2.
        \]
        Подставляя значения в формулу кривизны:
        \[
        k(t) = \frac{|[ r'(t), r''(t) ]|}{|r'(t)|^3} = \frac{|R|^2}{R^3} = \frac{1}{R}.
        \]
        Таким образом, радиус окружности равен:
        \[
        R = \frac{1}{k(t)}.
        \]
        
        \subsection*{№2 Нахождение вектора нормали}
        Нормальный вектор направлен перпендикулярно касательной к кривой в точке. Для его нахождения нужно определить направление вектора производной и нормализовать его. Используем формулы для нормали:
        \[
        v(t) = 
        \frac{\gamma'(t)}{|\gamma''(t)|}.
    , \quad 
        n(t) = \frac{v(t)}{|v(t)|}.
        \]
        Для кривой $\gamma(t)$ параметры $x(t)$ и $y(t)$ связаны производными. Формула кривизны также выражается как:
        \[
        k(t) = \frac{|\gamma', \gamma''|}{|\gamma'|^3}.
        \]
        И отсюда радиус определяется:
        \[
        R = \frac{1}{k(t)} = \frac{|\gamma'|^3}{|[\gamma', \gamma'']|}.
        \]
        Для двумерной кривой это упрощается до:
        \[
        R = \frac{(x'^2 + y'^2)^{\frac{3}{2}}}{|x'y'' - y'x''|}.
        \]
        
        \subsection*{№3 Уравнение соприкасающейся окружности}
        Соприкасающаяся окружность в каждой точке $\gamma(t)$ состоит из трёх компонентов:
        \begin{itemize}
            \item вектора положения кривой $\gamma(t)$;
            \item отрицательного нормального вектора $n(t)$, умноженного на радиус $R$;
            \item параметрической записи окружности радиуса $R$.
        \end{itemize}
        Формула окружности:
        \[
        R(t, \phi) = \gamma(t) - Rn(t) + r(\phi),
        \]
        где $r(\phi)$ — это параметрическое представление окружности:
        \[
        r(\phi) = (R \cos \phi, R \sin \phi).
        \]
        Подставляя выражение для радиуса и нормали, получаем итоговую формулу:
        \[
        R(t, \phi) = \gamma(t) - \frac{1}{k(t)} n(t) + \frac{1}{k(t)} (\cos \phi, \sin \phi).
        \]

       
        
    
\end{document}
