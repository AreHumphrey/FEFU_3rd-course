\documentclass{article}

\usepackage[T2A]{fontenc}
\usepackage[utf8]{inputenc} 
\usepackage[english,russian]{babel} 
\usepackage{graphicx} 
\usepackage{amsmath}
\usepackage{amssymb}
\usepackage{cancel}
\usepackage{amsfonts}
\usepackage{titlesec}
\usepackage{titling} 
\usepackage{geometry}
\usepackage{pgfplots}
\usepackage{esint}
\pgfplotsset{compat=1.9}
\usepackage{amsthm}


\titleformat{\section}
{\normalfont\Large\bfseries}{\arabic{section}}{1em}{}
\titleformat{\subsection}
{\normalfont\large\bfseries}{}{1em}{}


\setlength{\droptitle}{-3em} 
\title{\vspace{-1cm}Домашняя работа №8 \\ по дисциплине "Дифференциальная геометрия и топология"}
\author{Винницкая Дина Сергеевна}
\date{Группа: Б9122-02.03.01сцт}

\geometry{a4paper, margin=2cm}

\begin{document}
	
	\maketitle


        \subsection*{Условие задачи}
        Рассмотрим кривую, заданную параметрически:
        \[
        \gamma(t) = (t, \sin(t), \cos(t)).
        \]
        \section{Найти уравнение касательной в t = 0}

        \section*{Решение}
        Для нахождения уравнения касательной, найдем производную параметризации \( \gamma(t) \):
        \[
        \gamma'(t) = \left(1, \cos(t), -\sin(t)\right).
        \]
        
        Подставим значение параметра \( t = 0 \) в функцию \( \gamma(t) \) и её производную:
        \[
        \gamma(0) = (0, \sin(0), \cos(0)) = (0, 0, 1),
        \]
        \[
        \gamma'(0) = \left(1, \cos(0), -\sin(0)\right) = (1, 1, 0).
        \]
        
        Уравнение касательной в параметрическом виде задается формулой:
        \[
        x = x_0 + t \cdot \frac{dx}{dt}, \quad
        y = y_0 + t \cdot \frac{dy}{dt}, \quad
        z = z_0 + t \cdot \frac{dz}{dt}.
        \]
        
        Подставляя \( \gamma(0) = (0, 0, 1) \) и \( \gamma'(0) = (1, 1, 0) \), получаем:
        \[
        x = 0 + t \cdot 1, \quad y = 0 + t \cdot 1, \quad z = 1 + t \cdot 0.
        \]
        
        Запишем это в каноническом виде прямой:
        \[
        \frac{x - 1}{1} = \frac{y}{1} = \frac{z - 1}{0}
        \]
        

        \section{Найти длину дуги кривой \( \gamma(t) = (t, \sin(t), \cos(t)) \) на интервале \( t \in [0, 2\pi] \).}
        \section*{Решение}

        Для нахождения длины дуги кривой используем формулу:
        \[
        L(\gamma) = \int_0^{2\pi} \left|\gamma'(t)\right| \, dt,
        \]
        где \( \gamma'(t) \) — производная параметрической кривой.
        
        \paragraph{1. Найдём производную \( \gamma'(t) \):}
        \[
        \gamma'(t) = \left( \frac{d}{dt}t, \frac{d}{dt}\sin(t), \frac{d}{dt}\cos(t) \right) = \left( 1, \cos(t), -\sin(t) \right).
        \]
        
        \paragraph{2. Найдём длину вектора \( \gamma'(t) \):}
        \[
        \left|\gamma'(t)\right| = \sqrt{1^2 + \cos^2(t) + (-\sin(t))^2}.
        \]
        
        Упростим выражение, зная тригонометрическое тождество \( \cos^2(t) + \sin^2(t) = 1 \):
        \[
        \left|\gamma'(t)\right| = \sqrt{1 + \cos^2(t) + \sin^2(t)}
        \]
        
        \paragraph{3. Вычислим длину дуги \( L(\gamma) \):}
        Подставим значение \( \left|\gamma'(t)\right| = \sqrt{2} \) в формулу длины дуги:
        \[
        L(\gamma) = \int_0^{2\pi} \sqrt{1^2 + 1} \, dt.
        \]
        
        Так как \( \sqrt{2} \) является константой, она выносится за знак интеграла:
        \[
        L(\gamma) = \sqrt{2} \int_0^{2\pi} 1 \, dt.
        \]
        
        Рассчитаем интеграл:
        \[
        \int_0^{2\pi} 1 \, dt = \left[t\right]_0^{2\pi} = 2\pi - 0 = 2\pi.
        \]
        
        Подставляем результат:
        \[
        L(\gamma) = \sqrt{2} \cdot 2\pi = 2\sqrt{2}\pi.
        \]
        
        \subsection*{Ответ:}
        Длина дуги кривой \( \gamma(t) \) на интервале \( t \in [0, 2\pi] \) равна:
        \[
        L(\gamma) = 2\sqrt{2}\pi.
        \]


        \section{Задание}
        \[
        \text{Найти базис Френеля в } t = 0
        \]
        \[
        \left( \vec{v}, \vec{n}, \vec{b} \right)
        \]
        \[
        \vec{v} = \frac{\gamma'(t)}{\left|\gamma'(t)\right|}
        \]
        \[
        \vec{n} = \frac{\vec{v}'}{\left|\vec{v}'\right|}
        \]
        \[
        \vec{b} = \frac{[\vec{v}, \vec{n}]}{\left|[\vec{v}, \vec{n}]\right|}
        \]
        \section*{Решение}
        \paragraph{1. Найдём производную параметризации \(\gamma(t)\):}
        Производная вектора \(\gamma(t)\) задается как:
        \[
        \gamma'(t) = \left( \frac{d}{dt}t, \frac{d}{dt}\sin(t), \frac{d}{dt}\cos(t) \right).
        \]
        Выполняем вычисления:
        \[
        \gamma'(t) = (1, \cos(t), -\sin(t)).
        \]
        
        \paragraph{2. Найдём длину вектора \(\gamma'(t)\):}
        Длина вектора \(\gamma'(t)\) вычисляется по формуле:
        \[
        \left|\gamma'(t)\right| = \sqrt{1^2 + \cos^2(t) + \sin^2(t)}.
        \]
        Используя тригонометрическое тождество \( \cos^2(t) + \sin^2(t) = 1 \), получаем:
        \[
        \left|\gamma'(t)\right| = \sqrt{1 + 1} = \sqrt{2}.
        \]
        
        \paragraph{3. Найдём единичный касательный вектор \(\vec{v}\):}
        Касательный вектор определяется как:
        \[
        \vec{v} = \frac{\gamma'(t)}{\left|\gamma'(t)\right|}.
        \]
        Подставляя \(\gamma'(t)\) и \(\left|\gamma'(t)\right|\), получаем:
        \[
        \vec{v} = \left(\frac{1}{\sqrt{2}}, \frac{\cos(t)}{\sqrt{2}}, \frac{-\sin(t)}{\sqrt{2}}\right).
        \]
        
        \paragraph{4. Найдём производную касательного вектора \(\vec{v}'\):}
        Производная вектора \(\vec{v}\) задается как:
        \[
        \vec{v}' = \left(0, -\frac{\sin(t)}{\sqrt{2}}, -\frac{\cos(t)}{\sqrt{2}}\right).
        \]
        
        \paragraph{5. Найдём длину вектора \(\vec{v}'\):}
        Длина вектора \(\vec{v}'\) вычисляется по формуле:
        \[
        \left|\vec{v}'\right| = \sqrt{\left(-\frac{\sin(t)}{\sqrt{2}}\right)^2 + \left(-\frac{\cos(t)}{\sqrt{2}}\right)^2}.
        \]
        Подставляя тригонометрическое тождество:
        \[
        \left|\vec{v}'\right| = \sqrt{\frac{\sin^2(t)}{2} + \frac{\cos^2(t)}{2}} = \sqrt{\frac{\sin^2(t) + \cos^2(t)}{2}} = \sqrt{\frac{1}{2}} = \frac{1}{\sqrt{2}}.
        \]
        
        \paragraph{6. Найдём единичный нормальный вектор \(\vec{n}\):}
        Нормальный вектор определяется как:
        \[
        \vec{n} = \frac{\vec{v}'}{\left|\vec{v}'\right|}.
        \]
        Подставляя \(\vec{v}'\) и \(\left|\vec{v}'\right|\), получаем:
        \[
        \vec{n} = (0, -\sin(t), -\cos(t)).
        \]
        
        \paragraph{7. Найдём векторное произведение \([\vec{v}, \vec{n}]\):}
        Для вычисления бинормального вектора используем векторное произведение:
        \[
        [\vec{v}, \vec{n}] = \frac{1}{\sqrt{2}}(1, \cos(t), -\sin(t)).
        \]
        
        \paragraph{8. Найдём длину вектора \([\vec{v}, \vec{n}]\):}
        Длина вектора \([\vec{v}, \vec{n}]\) равна:
        \[
        \left|[\vec{v}, \vec{n}]\right| = \sqrt{\frac{1}{2}(1 + \cos^2(t) + \sin^2(t))}.
        \]
        Используя тригонометрическое тождество \( \cos^2(t) + \sin^2(t) = 1 \), получаем:
        \[
        \left|[\vec{v}, \vec{n}]\right| = 1.
        \]
        
        \paragraph{9. Найдём единичный бинормальный вектор \(\vec{b}\):}
        Бинормальный вектор определяется как:
        \[
        \vec{b} = \frac{[\vec{v}, \vec{n}]}{\left|[\vec{v}, \vec{n}]\right|}.
        \]
        Подставляя значения, получаем:
        \[
        \vec{b} = \frac{1}{\sqrt{2}}(1, \cos(t), -\sin(t)).
        \]
        
        \paragraph{10. Подставляем \(t = 0\):}
        Для \(t = 0\) вычисляем:
        \[
        \vec{n}(0) = (0, 0, -1), \quad
        \vec{b}(0) = \frac{1}{\sqrt{2}}(1, 1, 0).
        \]
        
        \subsection*{Ответ:}
        \[
        \vec{n}(0) = (0, 0, -1), \quad
        \vec{b}(0) = \frac{1}{\sqrt{2}}(1, 1, 0).
        \]

                    
        
       
                              
                
\end{document}
