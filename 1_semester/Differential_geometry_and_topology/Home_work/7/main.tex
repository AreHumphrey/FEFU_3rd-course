\documentclass{article}

\usepackage[T2A]{fontenc}
\usepackage[utf8]{inputenc} 
\usepackage[english,russian]{babel} 
\usepackage{graphicx} 
\usepackage{amsmath}
\usepackage{amssymb}
\usepackage{cancel}
\usepackage{amsfonts}
\usepackage{titlesec}
\usepackage{titling} 
\usepackage{geometry}
\usepackage{pgfplots}
\usepackage{esint}
\pgfplotsset{compat=1.9}
\usepackage{amsthm}


\titleformat{\section}
{\normalfont\Large\bfseries}{\arabic{section}}{1em}{}
\titleformat{\subsection}
{\normalfont\large\bfseries}{}{1em}{}


\setlength{\droptitle}{-3em} 
\title{\vspace{-1cm}Домашняя работа №7 \\ по дисциплине "Дифференциальная геометрия и топология"}
\author{Винницкая Дина Сергеевна}
\date{Группа: Б9122-02.03.01сцт}

\geometry{a4paper, margin=2cm}

\begin{document}
	
	\maketitle


        
        \section{Задание}
        Постройте непрерывную биекцию \( f : [0; 1) \rightarrow S^1 \) не являющуюся гомеоморфизмом.\\ 
        Не умоляя общности, можно считать, что \( S^1 \) — единичная окружность с центром в точке \( (0; 0) \):
        \[
        S^1 = \{ (x, y) \in \mathbb{R}^2 : \sqrt{x^2 + y^2} = 1 \}
        \]
        
        \( T_{\mathbb{R}}, T_{\mathbb{R}^2} \) — канонические топологии на \( \mathbb{R} \) и \( \mathbb{R}^2 \) соответственно:
        \[
        T_{\mathbb{R}x} = \{ U \cap [0; 1) \mid U \in T_{\mathbb{R}} \}
        \]
        \[
        T_{S^1} = \{ U \cap S^1 \mid U \in T_{\mathbb{R}^2} \}
        \]

        
        \section*{Решение}

        Рассмотрим отображение \( f : ([0; 1), T) \rightarrow (S^1, T_{S}) \), заданное следующим образом:
        \[
        f(t) = (\cos(2 \pi \cdot t), \; \sin(2 \pi \cdot t)),
        \]
        где \( f \) отображает отрезок \([0, 1)\) на единичную окружность \( S^1 \) в \( \mathbb{R}^2 \), и функция определяет точку окружности через угол \( 2\pi t \).
        
        \textbf{Обратное отображение}\\ \\
        Для любой точки \((x, y) \in S^1\) обратное отображение можно выразить через угол, определяемый функцией
        \[
        \text{Angle}(y, x) = 
        \begin{cases} 
        \arctan\left(\frac{y}{x}\right), & \text{если } x > 0, \; y \geq 0 \\
        \arctan\left(\frac{y}{x}\right) + 2 \pi, & \text{если } x > 0, \; y < 0 \\
        \arctan\left(\frac{y}{x}\right) + \pi, & \text{если } x < 0 \\
        \frac{\pi}{2}, & \text{если } x = 0, \; y > 0 \\
        \frac{3\pi}{2}, & \text{если } x = 0, \; y < 0 
        \end{cases}
        \]\\ \\
        Таким образом, обратное отображение \( f^{-1} \) определено как
        \[
        f^{-1}((x, y)) = \frac{\text{Angle}(y, x)}{2\pi}.
        \]\\ \\
        \textbf{1. Биективность}\\ \\
        Отображение \( f \) является биективным, так как каждому значению \( t \in [0, 1) \) соответствует уникальная точка на окружности \( S^1 \), и наоборот.\\ \\
        \textbf{2. Открытые множества и топология}\\ \\
        Поскольку \( f \) отображает \([0, 1)\) на \( S^1 \), нам нужно определить открытые множества на окружности.\\ \\
        Возьмем базисные множества для топологии в \( \mathbb{R}^2 \):
        \[
        \Sigma_{T_{\mathbb{R}^2}} = \{U(x_0, y_0, r) \mid \text{окрестности с центром } (x_0, y_0) \text{ и радиусом } r \} \; \text{— база топологии} \; T_{\mathbb{R}^2}.
        \] \\ \\
        Тогда множество \( \Sigma_{T_{S^1}} = \{S^1 \cap U : U \in \Sigma_{T_{\mathbb{R}^2}}\} \) образует базу топологии \( T_{S^1} \), заданной на \( S^1 \).\\ \\
        Каждое открытое множество \( D \in \Sigma_{T_{S^1}} \) — это дуга окружности, не содержащая своих граничных точек.\\ \\ Следовательно, при обратном отображении \( f^{-1}(D) = (a; b) \in [0; 1) \) будет интервалом, также принадлежащим \( T_{\mathbb{R}} \), поскольку интервал \( (a; b) \in T_{\mathbb{R}} \).\\ \\
        Таким образом, \( f \) является непрерывным. \\ \\
        Тогда \( f^{-1}(D) = (a; b) \in [0; 1) \) — интервал
        \((a; b) \in T_R \Rightarrow f^{-1}(D) = (a, b) \cap [0; 1) \in T_{R_X}\),
        \( f \) — непрерывна.\\ \\
        \textbf{3.}\\ \\
        \( V \in T_{R^2_{S^1}} \Rightarrow S^1 \setminus V \) замкнуто 
        \( \Rightarrow Fr \, S^1 \setminus V = Fr \, V \subset S^1 \setminus V \Rightarrow \) все граничные точки \( V \) лежат в его дополнении \(\Rightarrow \forall x \in D \exists U_x \in T_{R^2_{S^1}} : U_x \subset D \) \qquad (1) \\ \\
        Пусть \( U = [0; 0.5) \in T_{R_X} \)\\ \\
        \[ f^{-1}(f^{-1}(U)) = f(U) \]\\ \\
        \( x = f(0) \Rightarrow \forall U_x \in T_{R^2_{S^1}} \) существует \( x \in U_x \cap f(U) \Rightarrow U_x \cap f(U) \neq \emptyset \)\\ \\
        Существует \( \varepsilon > 0 : f(1 - \varepsilon) \in U_x \cap f((0; 1) \setminus U) \Rightarrow U_x \cap f((0; 1) \setminus U) \neq \emptyset \)\\ \\
        Следовательно, \( f(0) \) — граничная точка \( f(U) \) и \( f(0) \in f(U) \Rightarrow \) из (1) \( f(U) \) не является открытым множеством.\\ \\
        \[
        f^{-1} \text{ не является непрерывной функцией}
        \]\\ \\
        \( f \) — не гомеоморфизм.
        
    
        \section{Задание}
        \( T_X, T_Y \) — топологии на \( X, Y \).\\ \\
        Если ограничения отображения \( f : X \rightarrow Y \) на всех элементах покрытия \( \Gamma \) непрерывны, то \( f \) непрерывна.
        
        \begin{enumerate}
            \item \( X = [0; 2] \quad \Gamma = \{ V_1 = [0; 1], \; V_2 = (1; 2] \} \)
            \item \( X = [0; 2] \quad \Gamma = \{ V_1 = [0; 1], \; V_2 = [1; 2] \} \)
            \item \( X = \mathbb{R} \quad \Gamma = \{ \mathbb{Q}, \; \mathbb{R} \setminus \mathbb{Q} \} \)
        \end{enumerate}

        \section*{Решение}
        \section*{Теория}

        Пусть \( \Gamma = \{ V_i \} \) — покрытие множества \( X \), то есть объединение всех \( V_i \) равно \( X \): 
        \[
        \bigcup_{i \in I} V_i = X
        \]
            

        \[
        B \in T_Y \quad f^{-1}(B) = \{ x : f(x) \in B, \; x \in X \} = A
        \]
        
        \[
        f|_{V_i}^{-1}(B) = \{ x : f(x) \in B, \; x \in V_i \} = A_i \in T_{X_{V_i}}
        \]
        
        \[
        A_i = A \cap V_i
        \]
        
        \[
        A = \bigcup_{i \in I} A \cap V_i = \bigcup_{i \in I} A_i
        \]

        Задача сводится к вопросу: если для любого множества \( A \subset X \) выполняется, что \( A_i = A \cap V_i \in T_{X_{V_i}} \) для всех \( i \in I \), то справедливо ли, что \( A \in T_X \)?\\ \\ Иными словами, если на каждом элементе покрытия прообраз открытого множества \( B \) остаётся открытым, будет ли \( A \) открытым в топологии \( T_X \)?
        
        \subsection*{Первое достаточное условие непрерывности \( f \)}
        
        Первое условие заключается в следующем: если для любого множества \( A \subset X \) выполнение условия \( A_i = A \cap V_i \in T_{X_{V_i}} \) для всех \( i \in I \) влечёт, что \( A \in T_X \), то отображение \( f \) является непрерывным.\\ \\ Иными словами, прообраз любого открытого множества \( B \in T_Y \) остаётся открытым в \( T_X \), если на каждом элементе покрытия \( V_i \) прообраз также остаётся открытым.
        
        \subsection*{Второе достаточное условие непрерывности \( f \)}
        
        Второе достаточное условие можно получить, если использовать определение непрерывности через замкнутость.\\ \\ В этом случае прообраз любого замкнутого множества в \( Y \) должен оставаться замкнутым в \( X \). \\ \\
        Для любого \( B \in T_Y \) можно записать:
        \[
        f^{-1}(B) = \{ x : f(x) \in B, \; x \in X \} = A
        \]
        где
        \[
        f|_{V_i}^{-1}(B) = \{ x : f(x) \in B, \; x \in V_i \} = A_i; \quad A_i = A \cap V_i
        \]
        и, следовательно,
        \[
        A = \bigcup_{i \in I} A_i = \bigcup_{i \in I} (A \cap V_i)
        \] \\ \\ 
        Если для любого \( A \subset X \) выполнение условия \( A_i = A \cap V_i \in T_{X_{V_i}} \) для всех \( i \in I \) влечёт, что \( A \in T_X \), то отображение \( f \) будет непрерывным.
        
        \subsection*{Следствия и утверждения}
        На основе вышеизложенных условий можно сформулировать несколько утверждений.
        
        \subsection*{Утверждение 1}
        Если все \( V_i \) являются открытыми в \( T_X \) (то есть \( V_i \in T_X \) для всех \( i \in I \)), то \( f \) непрерывна.\\ \\
        \[
            \textbf{Доказательство} 
        \]
        \textbf{Необходимость} \\ \\
        Если \( V_i \in T_X \), то для любого множества \( A \subset X \), его прообразы \( A_i = A \cap V_i \in T_{X_{V_i}} \) будут открытыми в \( T_X \). Тогда
        \[
        A = \bigcup_{i \in I} A \cap V_i
        \]
        также будет открытым множеством в \( T_X \).\\ \\ Следовательно, \( f \) является непрерывным отображением. \\ \\
        \textbf{Достаточность} \\ \\
        Если \( A \in T_X \), то для любого множества \( V_i \) из покрытия \( \Gamma \) выполняется, что \( A \cap V_i = A_i \in T_{X_{V_i}} \), то есть прообраз любого открытого множества \( A \) остаётся открытым в топологии на каждом \( V_i \).\\ \\ Это означает, что \( f \) удовлетворяет первому достаточному условию непрерывности, следовательно, \( f \) — непрерывна.

        \subsection*{Утверждение 2}
        Предположим, что \( X \setminus V_i \in T_X \) для каждого \( i \in I \), и что покрытие \( \Gamma = \{ V_i \} \) конечно.\\ \\ Докажем, что при этих условиях отображение \( f \) будет непрерывным.
        \[
            \textbf{Доказательство} 
        \]
        Для каждого множества \( A \subset X \) определим его прообраз на элементах покрытия, как \( A_i = A \cap V_i \).\\ \\ Поскольку \( X \setminus V_i \) принадлежит топологии \( T_X \), для каждого \( i \) множество \( V_i \cap A_i \) также будет открытым в топологии \( T_{X_{V_i}} \).\\ \\ Следовательно, каждый прообраз \( A_i \) можно записать как разность:
        \[
        A_i = V_i \setminus (V_i \setminus A_i),
        \]
        где \( V_i \setminus A_i \) является замкнутым множеством, и, следовательно, \( V_i \cap A_i \) — открытое в \( T_{X_{V_i}} \).\\ \\
        Объединение всех \( A_i \) даёт нам множество \( A \):
        \[
        A = \bigcup_{i \in I} A_i = \bigcup_{i \in I} (A \cap V_i),
        \]
        и, поскольку покрытие \( \{ V_i \} \) конечно, это объединение также будет конечным. \\ \\ Таким образом, множество \( A \) является конечным объединением замкнутых множеств в \( T_X \), что делает его замкнутым.\\ \\
        Теперь рассмотрим обратное: если \( X \setminus A \in T_X \), тогда для каждого \( i \in I \) множество \( V_i \cap A_i = A_i \) также замкнуто.\\ \\ Из второго достаточного условия получаем, что \textbf{f - непрерывна}


        \subsection*{2.3 Доказательство}
        Рассмотрим топологию \( T_X \), которая индуцирована из канонической топологии на \( \mathbb{R} \) в \( X \).

        \begin{enumerate}
            \item Зададим множество \( Y = X \) и функцию \( f(x) \), определённую по следующему правилу:
            \[
            f(x) = \begin{cases} 
            1 - x, & x \in [0; 1] \\ 
            x, & x \in (1; 2]
            \end{cases}
            \]
            При этом функция \( f \) действует следующим образом:
            \begin{itemize}
                \item Ограничение \( f \) на \( V_2 \) является тождественным отображением \( id : V_2 \rightarrow X \), то есть \( f|_{V_2} = id \), и оно является непрерывным.
                \item Ограничение \( f \) на \( V_1 \) также непрерывно, так как \( f|_{V_1} : V_1 \rightarrow X \) сохраняет топологические свойства в этом подмножестве.
            \end{itemize}
            Однако, если рассматривать \( f \) на всём \( X \), то она уже не является непрерывной. Это можно проиллюстрировать на примере множества \( [0; 0.5) \), которое принадлежит топологии \( T_X \).\\ \\ Обратный образ этого множества при \( f \) будет равен \( f^{-1}([0; 0.5)) = (0.5; 1] \), и это множество не принадлежит \( T_X \). Таким образом, \( f : X \rightarrow Y \) не является непрерывной в общем случае.
            \item Второй случай непрерывности рассмотрен на основе утверждения 2. Поскольку \( X \setminus V_i \in T_X \) для каждого \( i \in I \), и если покрытие \( V_i \) конечно, то \( f \) сохраняет непрерывность. \\ \\Однако, в данном случае это условие не выполняется для всех элементов, и потому \( f \) не является непрерывной.
        
            \item Теперь рассмотрим случай с антидискретными топологиями на рациональных и иррациональных числах.
            \end{enumerate}
            \[
        \forall U \not\in \{\emptyset, Q\} \quad U \in T_Q \Rightarrow U = V \cap Q
        \]
        
        Пусть \( V = (\min(Q) - \varepsilon_1; \max(Q) + \varepsilon_2) \), где \( \varepsilon_1 \) и \( \varepsilon_2 \) — произвольные положительные числа, расширяющие границы \( Q \).
        
        \[
        \forall \varepsilon_1, \varepsilon_2 \quad V \cap Q \neq U
        \]\\
        То есть, для любых значений \( \varepsilon_1 \) и \( \varepsilon_2 \), пересечение \( V \) и \( Q \) не совпадает с \( U \).\\ \\
        Аналогично рассуждаем для \( T_{R \setminus Q} \), что позволяет заключить, что обе топологии \( T_Q \) и \( T_{R \setminus Q} \) являются антидискретными топологиями, где открытыми множествами являются только \( \emptyset \) и \( Q \) (или \( R \setminus Q \) соответственно).\\ \\
        Таким образом, для любого множества \( B \in Y \) выполняется следующее:
        \[
        f|_{Q}^{-1}(B) \in \{\emptyset, Q\} \quad \text{и} \quad f|_{R \setminus Q}^{-1}(B) \in \{\emptyset, R \setminus Q\}
        \]
        Это означает, что прообразы множества \( B \) при ограничении отображения \( f \) на \( Q \) и на \( R \setminus Q \) могут быть либо пустыми, либо соответствующими множествами \( Q \) или \( R \setminus Q \).\\ \\
        Тогда полный прообраз \( f^{-1}(B) \) принимает значения из множества \(\{\emptyset, Q, R \setminus Q, R\}\), которое состоит из пустого множества, множества \( Q \), множества \( R \setminus Q \) и всего пространства \( R \).\\ \\
        Однако, так как \( Q \) и \( R \setminus Q \) не принадлежат топологии \( T_R \), это означает, что эти множества не являются открытыми в \( T_R \).\\ \\
        Следовательно, в общем случае функция \( f \) не является непрерывной, так как прообразы открытых множеств не обязательно остаются открытыми.
        
            
            
                
        

        
                
                               
                
\end{document}
