\documentclass{article}

\usepackage[T2A]{fontenc}
\usepackage[utf8]{inputenc} 
\usepackage[english,russian]{babel} 
\usepackage{graphicx} 
\usepackage{amsmath}
\usepackage{amssymb}
\usepackage{cancel}
\usepackage{amsfonts}
\usepackage{titlesec}
\usepackage{titling} 
\usepackage{geometry}
\usepackage{pgfplots}
\usepackage{esint}
\pgfplotsset{compat=1.9}
\usepackage{amsthm}


\titleformat{\section}
{\normalfont\Large\bfseries}{\arabic{section}}{1em}{}
\titleformat{\subsection}
{\normalfont\large\bfseries}{}{1em}{}


\setlength{\droptitle}{-3em} 
\title{\vspace{-1cm}Домашняя работа №13 \\ по дисциплине "Дифференциальная геометрия и топология"}
\author{Винницкая Дина Сергеевна}
\date{Группа: Б9122-02.03.01сцт}

\geometry{a4paper, margin=2cm}

\begin{document}
	
	\maketitle


        \subsection*{Условие задачи}
        Вычислить коэффициент \(I\), где \(I\) — параметр, проверяющий свойства цилиндра.

            
        \section*{Решение}
       \subsection*{Параметризация цилиндра}

Цилиндр параметризуется следующим образом:
\[
\vec{R}(u, v) = \begin{pmatrix}
R \cdot \cos u, \\
R \cdot \sin u, \\
v
\end{pmatrix},
\]
где:
\begin{itemize}
    \item \(R\) — радиус цилиндра;
    \item \(u\) — угол, задающий положение точки на окружности (основе цилиндра);
    \item \(v\) — вертикальная координата вдоль оси цилиндра.
\end{itemize}

\subsection*{Частные производные параметризации}

Рассчитаем частные производные параметризации по параметрам \(u\) и \(v\).

\paragraph{Производная по \(u\):}
\[
\frac{\partial \vec{R}}{\partial u} = \begin{pmatrix}
\frac{\partial}{\partial u}(R \cdot \cos u), \\
\frac{\partial}{\partial u}(R \cdot \sin u), \\
\frac{\partial}{\partial u}(v)
\end{pmatrix}
= \begin{pmatrix}
-R \cdot \sin u, \\
R \cdot \cos u, \\
0
\end{pmatrix}.
\]

\paragraph{Производная по \(v\):}
\[
\frac{\partial \vec{R}}{\partial v} = \begin{pmatrix}
\frac{\partial}{\partial v}(R \cdot \cos u), \\
\frac{\partial}{\partial v}(R \cdot \sin u), \\
\frac{\partial}{\partial v}(v)
\end{pmatrix}
= \begin{pmatrix}
0, \\
0, \\
1
\end{pmatrix}.
\]

\subsection*{Коэффициенты метрики}

С помощью частных производных определим коэффициенты первой фундаментальной формы.

\paragraph{Коэффициент \(E\):}
\[
E = \left\|\frac{\partial \vec{R}}{\partial u}\right\|^2 = (-R \cdot \sin u)^2 + (R \cdot \cos u)^2 + 0^2.
\]
Учитывая тригонометрическую идентичность \(\sin^2 u + \cos^2 u = 1\), получаем:
\[
E = R^2 \cdot (\sin^2 u + \cos^2 u) = R^2.
\]

\paragraph{Коэффициент \(F\):}
\[
F = \frac{\partial \vec{R}}{\partial u} \cdot \frac{\partial \vec{R}}{\partial v} = 
(-R \cdot \sin u) \cdot 0 + (R \cdot \cos u) \cdot 0 + 0 \cdot 1.
\]
Отсюда:
\[
F = 0.
\]

\paragraph{Коэффициент \(G\):}
\[
G = \left\|\frac{\partial \vec{R}}{\partial v}\right\|^2 = 0^2 + 0^2 + 1^2 = 1.
\]

\subsection*{Итоговый результат}

Коэффициенты метрики цилиндра:
\[
E = R^2, \quad F = 0, \quad G = 1.
\]

Таким образом, параметризация цилиндра выражается в следующем виде:
\[
\vec{R}(u, v) = \begin{pmatrix}
R \cdot \cos u, \\
R \cdot \sin u, \\
v
\end{pmatrix},
\]
где коэффициенты \(E\), \(F\), \(G\) подтверждают корректность параметризации.
                
                
\end{document}
