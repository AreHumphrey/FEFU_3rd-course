\documentclass{article}

\usepackage[T2A]{fontenc}
\usepackage[utf8]{inputenc} 
\usepackage[english,russian]{babel} 
\usepackage{graphicx} 
\usepackage{amsmath}
\usepackage{amssymb}
\usepackage{cancel}
\usepackage{amsfonts}
\usepackage{titlesec}
\usepackage{titling} 
\usepackage{geometry}
\usepackage{pgfplots}
\usepackage{esint}
\pgfplotsset{compat=1.9}
\usepackage{amsthm}


\titleformat{\section}
{\normalfont\Large\bfseries}{\arabic{section}}{1em}{}
\titleformat{\subsection}
{\normalfont\large\bfseries}{}{1em}{}


\setlength{\droptitle}{-3em} 
\title{\vspace{-1cm}Домашняя работа №5 \\ по дисциплине "Дифференциальная геометрия и топология"}
\author{Винницкая Дина Сергеевна}
\date{Группа: Б9122-02.03.01сцт}

\geometry{a4paper, margin=2cm}

\begin{document}
	
	\maketitle


        \section{Задание}

        Найти \( Int(0, 1) \) в топологии Зарисского. 
        
        Пусть:
        \[
        X = \mathbb{R}
        \]
        \[
        T = \text{все множества, являющиеся дополнениями конечных подмножеств, или пустое множество}
        \]
        \subsection{Решение:}
    
        Рассмотрим множество:
        \[
        A = \mathbb{R} \setminus ((-\infty, 0] \cup [1, +\infty))
        \]
        Предположим, что существует такая точка \( x \in A \), что найдётся её окрестность \( U_x \subset Int(A) \), где \( U_x = \mathbb{R} \setminus V \), при этом \(V\) — конечное множество.
        
        \[
        V = \mathbb{R} \setminus U_x
        \]
        \[
        \mathbb{R} \setminus A \subset V
        \]
        \[
        V\supset \mathbb{R} \setminus A = (-\infty, 0] \cup [1, +\infty)
        \]
        Следовательно, \(V\) оказывается бесконечным, что противоречит предположению о его конечности.
        Таким образом, предположение оказалось неверным.
        Следовательно, \( Int(0, 1) = \emptyset \).


        
        \section{Задание}
        В \( \mathbb{R}, \, \mathcal{T}_{\text{канонич}} \): найти \( Cl([0, 1]), Cl(\mathbb{Q}), Cl(\mathbb{R} \setminus \mathbb{Q}), Cl(\{a\}) \) в \( \xi \):

\[
\xi = \begin{pmatrix}
X = \{a, b, c, d\} \\
T_X = \{\emptyset, X, \{a\}, \{b\}, \{a, b\}, \{a, b, c\}, \{a, b\}\}
\end{pmatrix}
\]

\section*{Решение}

Замыканием множества \( A \) называется совокупность всех его точек прикосновения, и обозначается как \( Cl_x(A) \).

\[
Cl_x(A) = \{x \in X \,|\, \forall U_x \quad U_x \cap A \neq \emptyset\}
\]
\[
Cl(A) = \bigcap F_i, \quad F_i - \text{замкнуто и} \quad F_i \supset A
\]
Каноническая топология на \( \mathbb{R} \) определяется как топология, в основе которой лежат открытые интервалы, то есть:

\[
U \in \mathcal{T} \quad \Longleftrightarrow \quad U = \emptyset \quad \text{или} \quad \forall x \in U \, \exists V : V = \{x \,|\, |x - x_0| < \varepsilon\} : V \subseteq U
\]
\[
\mathbb{R} \setminus A = (-\infty, 0) \cup (1; +\infty) \quad \text{– открытое}
\]
\[
\Rightarrow [0, 1] \quad \text{– замкнутое, и является наименьшим замкнутым множеством, содержащим} \, A
\]
\[
Cl([0, 1]) = [0, 1]
\]

\subsection*{Замыкание \( Cl(\mathbb{Q}) \)}
\[
\forall x \in \mathbb{R} \, \forall U_x \, \text{в каждой окрестности существуют рациональные точки}
\]
\[
\text{Следовательно, любая точка} \, \mathbb{R} \, \text{– точка прикосновения множества} \, \mathbb{Q}
\]
\[
Cl(\mathbb{Q}) = \mathbb{R}
\]

\subsection*{Замыкание \( Cl(\mathbb{R} \setminus \mathbb{Q}) \)}
\[
\forall x \in (\mathbb{R} \setminus \mathbb{Q}) \, \forall \, U_x, \, \text{в каждой окрестности существуют рациональные точки}
\]
\[
\text{Следовательно, любая точка множества} \, \mathbb{R} \, \text{– точка прикосновения множества} \, \mathbb{R} \setminus \mathbb{Q}
\]
\[
Cl(\mathbb{R} \setminus \mathbb{Q}) = \mathbb{R}
\]

\subsection*{Замыкание \( Cl(\{a\}) \)}
\[
\{a, c, d\} = X \setminus \{d\}
\]
\[
Cl(\{a\}) = \{a, c, d\}
\]

        
        \section*{Задание}
        Показать, что множество \( A \) замкнуто тогда и только тогда, когда его граница \(A\) содержится в самом множестве \(A\), то есть:
        \[
        A \, \text{замкнуто} \quad \Longleftrightarrow \quad \partial A \subseteq A
        \]
        
        \section*{Решение}
Граница \( A = \partial A \)\\\\
\textbf{Необходимость}\\\\
\( A \) — замкнутое множество \(\Rightarrow X \setminus A \in T \Rightarrow \forall x \in X \setminus A \, \exists U_x = X \setminus A : U_x \cap A = \emptyset\)
\(\Rightarrow x\) не является граничной точкой для \( A \) \(\forall x \in X \setminus A \Rightarrow \partial A \subseteq A\)\\\\
\textbf{Достаточность}\\\\
\(\partial A \subseteq A \Rightarrow \exists x \in A : \forall U_x, \, U_x \cap A \neq \emptyset \, \lor \, U_x \cap V \neq \emptyset, \, V = X \setminus A\)\\
Поскольку \(\partial A \nsubseteq V\), то \(\forall x \in V \, \exists U_x \subseteq V\), так что \( U_x \cap A = \emptyset \)\\
Следовательно, \( Cl(A) = A \)
Так как \( Cl \) — замкнуто, то и \( A \) — замкнутое

        
               
        
\end{document}
