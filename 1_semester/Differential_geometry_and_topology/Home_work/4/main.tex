\documentclass{article}

\usepackage[T2A]{fontenc}
\usepackage[utf8]{inputenc} 
\usepackage[english,russian]{babel} 
\usepackage{graphicx} 
\usepackage{amsmath}
\usepackage{amssymb}
\usepackage{cancel}
\usepackage{amsfonts}
\usepackage{titlesec}
\usepackage{titling} 
\usepackage{geometry}
\usepackage{pgfplots}
\usepackage{esint}
\pgfplotsset{compat=1.9}
\usepackage{amsthm}


\titleformat{\section}
{\normalfont\Large\bfseries}{\arabic{section}}{1em}{}
\titleformat{\subsection}
{\normalfont\large\bfseries}{}{1em}{}


\setlength{\droptitle}{-3em} 
\title{\vspace{-1cm}Домашняя работа №4 \\ по дисциплине "Дифференциальная геометрия и топология"}
\author{Винницкая Дина Сергеевна}
\date{Группа: Б9122-02.03.01сцт}

\geometry{a4paper, margin=2cm}

\begin{document}
	
	\maketitle

        \section{Задание} 

     \begin{enumerate}
        \item Найти примеры и доказать, что множества, открытые в подпространстве, не обязательно открыты в объемлющем пространстве.
        \item Доказать, что \( F \) замкнуто в подпространстве \( A \subseteq X \), тогда и только тогда, когда \( F = A \cap E \), где \( E \) - замкнуто в \( X \).
    \end{enumerate}


        \section*{Найти примеры и доказать, что множества, открытые в подпространстве, не обязательно открыты в объемлющем пространстве.} 
        

        \subsection*{Решение}
        Рассмотрим следующее определение подпространства:
        \[
        \text{Пусть} \quad (X, T) - \text{топологическое пространство}.
        \]
        \[
        A \subseteq X - \text{произвольное подмножество}.
        \]
        Множество \( T_A \) представляет собой коллекцию множеств вида \( \{A \cap U \, | \, U \in T \} \), 
        а пару \( (A, T_A) \) называют подпространством топологического пространства \( (X, T) \).
        Топология \( T_A \) называется топологией, индуцированной на \( A \) топологическим пространством \( (X, T) \).

        
        \[
        X = \{a, b, c, d\}
        \]
        \[
        T_X = \{\{a, b\}, \{c, d\}, \varnothing, X\}
        \]
        \[
        A = \{b, c\}
        \]
        \[
        T_A = \{\{b\}, \{c\}, \varnothing, A\}
        \]
        \[
        \{b, c\} \notin T_X
        \]
        Следовательно, множество \( \{b, c\} \), которое открыто в подпространстве, не является открытым в \( (X, T) \). 
        
        \[
        \square
        \]
        
        Теперь рассмотрим этот же факт в более формальной записи:
        
        Пусть \( (X, T) \) - топологическое пространство. 
        
        Пусть \( A \subseteq X \), но при этом \( A \notin T_A \).
        
        Тогда множество \( A \cap U = X \) не будет элементом \( T_X \).

        \[
        \square
        \]

        \section*{  Доказать, что \( F \) замкнуто в подпространстве \( A \subseteq X \), тогда и только тогда, когда \( F = A \cap E \), где \( E \) - замкнуто в \( X \).} 

        \subsection*{Решение}

        
        \textbf{Необходимое условие:} 
        
        Пусть \( F \) замкнуто в \( A \), то есть существует \( L \in T_A \) такое, что \( F = A \setminus L \).
        
        Так как \( L = A \cap U \in T_X \) (по свойствам подпространства), имеем:
        
        \[
        F = A \setminus (A \cap U) = A \cap (X \setminus U) = A \cap E, \, где \, E = X \setminus U.
        \]
        Так как \( U \in T_X \), то \( E \) замкнуто в \( T_X \).
        
        \[
        A \setminus (A \cap U) = A \cap (X \setminus U)
        \]
        \[
        A \cap B = \{x \, | \, x \in A \text{ и } x \in B \}
        \]
        \[
        A \setminus B = \{x \, | \, x \in A \text{ и } x \notin B \}
        \]
        \[
        A \subseteq X \Longrightarrow E \in X.
        \]
        
        \[
        A \cap (A \cap U) = \{x \, | \, x \in A, x \in (A \cap B)\} = \{x \, | \, x \in A, x \notin B\}
        \]
        \[
        A \cap (X \setminus U) = \{x \, | \, x \in A, x \in (X \setminus B)\} = \{x \, | \, x \in A, x \notin B\}
        \]
        
        \textbf{Достаточное условие:}\\
        
        Пусть \( F = A \cap E \), где \( E \) замкнуто в \( T_X \) и является дополнением к открытому множеству \( U \).
        
        Для того чтобы \( F \) было замкнутым в \( A \), необходимо, чтобы \( F = A \setminus L \), где \( L \in T_A \).
        
        \[
        F = A \cap E = A \cap (X \setminus U) = A \setminus (A \cap U) = A \setminus L
        \]
        \[
        L = A \cap U \, \text{и} \, является элементом топологии подпространства по определению.
        \]
        Таким образом, \( F \) является дополнением к открытому множеству в подпространстве, следовательно, оно замкнуто.
        
        \[
        \square
        \]

        
               
        
\end{document}
