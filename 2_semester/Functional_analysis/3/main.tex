\documentclass{article}

\usepackage[T2A]{fontenc}
\usepackage[utf8]{inputenc} 
\usepackage[english,russian]{babel} 
\usepackage{graphicx} 
\usepackage{amsmath}
\usepackage{amssymb}
\usepackage{cancel}
\usepackage{amsfonts}
\usepackage{amsthm}
\usepackage{titlesec}
\usepackage{titling} 
\usepackage{geometry}
\usepackage{pgfplots}
\usepackage{esint}
\pgfplotsset{compat=1.9}
\usepackage{amsthm}
\usepackage{array} 
\usepackage{longtable}
\usepackage{booktabs} 
\usepackage{xcolor}


\titleformat{\section}
{\normalfont\Huge\bfseries}{\arabic{section}}{1em}{}
\titleformat{\subsection}
{\normalfont\Large\bfseries}{}{1em}{}

\setlength{\droptitle}{-3em} 
\title{\vspace{-1cm}ИДЗ №3 \\ по дисциплине "Функциональный анализ"}
\author{Винницкая Дина Сергеевна}
\date{Группа: Б9122-02.03.01сцт}

\geometry{a4paper, margin=2cm}

\usepackage{setspace}
\onehalfspacing  

\renewcommand{\baselinestretch}{1.6} 

\begin{document}
	
	\maketitle

\subsection{Задание 10.8. (a)}
$$f(x) = \int_0^1 \sqrt{t} \cdot x(t^2)\, dt,$$
 $$X = \left[ 0,1 \right];$$


\subsection{Решение}

Рассмотрим функционал
\[
f(x) = \int_0^1 \sqrt{t} \cdot x(t^2)\, dt,
\]
заданный на пространстве $C[0,1]$ — пространстве непрерывных функций на отрезке $[0,1]$ с нормой супремума $\|x\|_\infty = \sup_{t \in [0,1]} |x(t)|$.\\\\
Функция $x(t^2)$ определена корректно, поскольку при $t \in [0,1]$ имеем $t^2 \in [0,1]$, а значит аргумент функции $x$ лежит в области определения.\\ \\
Композиция непрерывной функции $x(t)$ с непрерывной функцией $t^2$ остаётся непрерывной, поэтому $x(t^2)$ — непрерывна на $[0,1]$. Функция $\sqrt{t}$ также непрерывна на $[0,1]$, включая $t = 0$.\\ \\ Следовательно, произведение $\sqrt{t} \cdot x(t^2)$ — непрерывная функция на $[0,1]$, а значит интеграл определён как обычный определённый интеграл Римана.\\ \\
Покажем теперь, что функционал $f$ линеен и непрерывен. Линейность очевидна, так как оператор интегрирования и умножение на фиксированную функцию сохраняют линейность.\\ \\ Проверим ограниченность функционала. По модулю:
\[
|f(x)| = \left| \int_0^1 \sqrt{t} \cdot x(t^2) \, dt \right| \leq \int_0^1 \sqrt{t} \cdot |x(t^2)| \, dt.
\]
Так как $x \in C[0,1]$, то $|x(t^2)| \leq \|x\|_\infty$ для всех $t \in [0,1]$. \\ \\ Получаем:
\[
|f(x)| \leq \|x\|_\infty \cdot \int_0^1 \sqrt{t} \, dt = \|x\|_\infty \cdot \left[\frac{2}{3}t^{3/2}\right]_0^1 = \frac{2}{3} \|x\|_\infty.
\]
\[
\|f\| = \frac{2}{3}
\]



\end{document}



\end{document}
