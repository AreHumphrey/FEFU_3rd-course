\documentclass{article}

\usepackage[T2A]{fontenc}
\usepackage[utf8]{inputenc} 
\usepackage[english,russian]{babel} 
\usepackage{graphicx} 
\usepackage{amsmath}
\usepackage{amssymb}
\usepackage{cancel}
\usepackage{amsfonts}
\usepackage{amsthm}
\usepackage{titlesec}
\usepackage{titling} 
\usepackage{geometry}
\usepackage{pgfplots}
\usepackage{esint}
\pgfplotsset{compat=1.9}
\usepackage{amsthm}
\usepackage{array} 
\usepackage{longtable}
\usepackage{booktabs} 
\usepackage{xcolor}


\titleformat{\section}
{\normalfont\Huge\bfseries}{\arabic{section}}{1em}{}
\titleformat{\subsection}
{\normalfont\Large\bfseries}{}{1em}{}

\setlength{\droptitle}{-3em} 
\title{\textbf{ИДЗ №1 \\ по дисциплине "Функциональный анализ"}}
\author{Винницкая Дина Сергеевна}
\date{Группа: Б9122-02.03.01сцт}

\geometry{a4paper, margin=2cm}

\usepackage{setspace}
\onehalfspacing  

\renewcommand{\baselinestretch}{1.6} 

\begin{document}
	
	\maketitle

\subsection{Задание 1.20. (д)} Построить шары $B[0, 1]$ в пространстве $\mathbb{R}^3$, если для $x = (\xi_1, \xi_2, \xi_3) \in \mathbb{R}^3$ нормы определены следующим образом:
 $$\|x\| = \sqrt{4\xi_1^2 + \dfrac{1}{9}\xi_2^2 + \xi_3^2}.$$


\subsection{Решение}

Пусть \( x = (\xi_1, \xi_2, \xi_3) \in \mathbb{R}^3 \). Тогда шар \( B[0,1] \) при норме из задания определяется как множество всех \( x \), для которых выполняется неравенство:

\[
\|x\| = \sqrt{4|\xi_1|^2 + \frac{1}{9}|\xi_2|^2 + |\xi_3|^2}
\]
то есть:

\[
\sqrt{4\xi_1^2 + \frac{1}{9}\xi_2^2 + \xi_3^2} \leq 1.
\]
\begin{center}
    \includegraphics[width=0.6\textwidth]{fu_1.png}
    
    \textit{Рисунок: График эллипсоида $\sqrt{4x^2 + \frac{1}{9}y^2 + z^2} \leq 1$}
\end{center}
Это множество представляет собой эллипсоид, сжатый по оси $\xi_1$ (в 2 раза), растянутый по оси $\xi_2$ (в 3 раза), и со стандартной масштабировкой по $\xi_3$.







\end{document}



\end{document}
