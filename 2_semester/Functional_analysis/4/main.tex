\documentclass{article}

\usepackage[T2A]{fontenc}
\usepackage[utf8]{inputenc} 
\usepackage[english,russian]{babel} 
\usepackage{graphicx} 
\usepackage{amsmath}
\usepackage{amssymb}
\usepackage{cancel}
\usepackage{amsfonts}
\usepackage{amsthm}
\usepackage{titlesec}
\usepackage{titling} 
\usepackage{geometry}
\usepackage{pgfplots}
\usepackage{esint}
\pgfplotsset{compat=1.9}
\usepackage{amsthm}
\usepackage{array} 
\usepackage{longtable}
\usepackage{booktabs} 
\usepackage{xcolor}


\titleformat{\section}
{\normalfont\Huge\bfseries}{\arabic{section}}{1em}{}
\titleformat{\subsection}
{\normalfont\Large\bfseries}{}{1em}{}

\setlength{\droptitle}{-3em} 
\title{\textbf{ИДЗ №4 \\ по дисциплине "Функциональный анализ"}}
\author{Винницкая Дина Сергеевна}
\date{Группа: Б9122-02.03.01сцт}

\geometry{a4paper, margin=2cm}

\usepackage{setspace}
\onehalfspacing  

\renewcommand{\baselinestretch}{1.6} 

\begin{document}
	
	\maketitle

\subsection{Задание 8.33. (a)} Найти ортогональную проекцию элемента
\[
x_0 = \left\{ \frac{1}{k} \right\}_{k=1}^{\infty} \in \ell_2
\]
на подпространство $L$, а также расстояния $\rho(x_0, L)$ и $\rho(x_0, L^\perp)$, если
$$
    L = \left\{ x = \{\xi_k\} \in \ell_2 : \xi_1 - 3\xi_3 + \xi_5 = 0 \right\}.
    $$



\subsection{Решение}


\textbf{Шаг 1. Определяем направление ортогонального дополнения \(L^\perp\).}\\ \\
Подпространство $L$ задано одним линейным уравнением — это значит, что $L$ является гиперплоскостью в $\ell_2$, перпендикулярной некоторому вектору $a$. Этот вектор получается из коэффициентов уравнения:
\[
\xi_1 - 3\xi_3 + \xi_5 = 0 \quad \Rightarrow \quad a = (1, 0, -3, 0, 1, 0, 0, \dots)
\]
Вектор $a$ задаёт линейную форму, ортогональную к $L$.\\ \\ Таким образом,
\[
L^\perp = \text{span}(a)
\]
— одномерное подпространство, порождённое вектором $a$.
\vspace{0.7em}
\textbf{Шаг 2. Строим ортогональную проекцию элемента \(x_0\) на \(L^\perp\).}\\ \\
Чтобы найти проекцию вектора $x_0$ на прямую, порождённую $a$, используем стандартную формулу проекции:
\[
P_{L^\perp} x_0 = \frac{\langle x_0, a \rangle}{\langle a, a \rangle} \cdot a.
\]
В числителе — скалярное произведение $x_0$ и $a$:
\[
\langle x_0, a \rangle = 1 \cdot \frac{1}{1} + (-3) \cdot \frac{1}{3} + 1 \cdot \frac{1}{5} = 1 - 1 + \frac{1}{5} = \frac{1}{5}.
\]
В знаменателе — квадрат нормы вектора $a$:
\[
\langle a, a \rangle = 1^2 + (-3)^2 + 1^2 = 1 + 9 + 1 = 11.
\]
Тогда проекция:
\[
P_{L^\perp} x_0 = \frac{1}{5 \cdot 11} \cdot a = \frac{1}{55} \cdot (1, 0, -3, 0, 1, 0, 0, \dots).
\]

\vspace{0.7em}
\textbf{Шаг 3. Вычисляем проекцию \(x_0\) на само подпространство \(L\).} \\ \\Ортогональная проекция на $L$ получается вычитанием проекции на $L^\perp$ из вектора $x_0$:
\[
P_L x_0 = x_0 - P_{L^\perp} x_0 = x_0 - \frac{1}{55} \cdot a.
\]
Явно:
\[
P_L x_0 = \left(1, \frac{1}{2}, \frac{1}{3}, \frac{1}{4}, \frac{1}{5}, \dots\right)
- \left( \frac{1}{55}, 0, -\frac{3}{55}, 0, \frac{1}{55}, 0, \dots \right).
\]

\vspace{0.7em}
\textbf{Шаг 4. Находим расстояние от \(x_0\) до подпространства \(L\).} \\ \\
Так как $P_{L^\perp} x_0$ — это компонент вектора, перпендикулярный $L$, то его длина — это расстояние от $x_0$ до $L$:
\[
\rho(x_0, L) = \left\| P_{L^\perp} x_0 \right\| = \left\| \frac{1}{55} \cdot a \right\| = \frac{1}{55} \cdot \|a\| = \frac{1}{55} \cdot \sqrt{11} = \frac{\sqrt{11}}{55} = \frac{1}{5\sqrt{11}}.
\]
\vspace{0.7em}
\textbf{Шаг 5. Находим расстояние от \(x_0\) до \(L^\perp\).} \\ \\
Это расстояние равно длине проекции $x_0$ на подпространство $L$, и его можно найти через теорему Пифагора:
\[
\|x_0\|^2 = \sum_{k=1}^{\infty} \frac{1}{k^2} = \frac{\pi^2}{6},
\]
\[
\left\| P_{L^\perp} x_0 \right\|^2 = \left( \frac{1}{55} \right)^2 \cdot \|a\|^2 = \frac{1}{3025} \cdot 11 = \frac{11}{3025} = \frac{1}{275}.
\]
Следовательно:
\[
\rho(x_0, L^\perp) = \sqrt{ \|x_0\|^2 - \|P_{L^\perp} x_0\|^2 } = \sqrt{ \frac{\pi^2}{6} - \frac{1}{275} }.
\]
\vspace{1em}
\textbf{Итог:}
\begin{itemize}
    \item Ортогональная проекция на $L$:
    \[
    P_L x_0 = x_0 - \frac{1}{55}(1, 0, -3, 0, 1, 0, \dots)
    \]
    \item Расстояние до $L$:
    \[
    \rho(x_0, L) = \frac{1}{5\sqrt{11}}
    \]
    \item Расстояние до $L^\perp$:
    \[
    \rho(x_0, L^\perp) = \sqrt{ \frac{\pi^2}{6} - \frac{1}{275} }
    \]
\end{itemize}






\end{document}


