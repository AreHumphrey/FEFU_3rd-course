\documentclass{article}

\usepackage[T2A]{fontenc}
\usepackage[utf8]{inputenc} 
\usepackage[english,russian]{babel} 
\usepackage{graphicx} 
\usepackage{amsmath}
\usepackage{amssymb}
\usepackage{cancel}
\usepackage{amsfonts}
\usepackage{amsthm}
\usepackage{titlesec}
\usepackage{titling} 
\usepackage{geometry}
\usepackage{pgfplots}
\usepackage{esint}
\pgfplotsset{compat=1.9}
\usepackage{amsthm}
\usepackage{array} 
\usepackage{longtable}
\usepackage{booktabs} 
\usepackage{xcolor}


\titleformat{\section}
{\normalfont\Huge\bfseries}{\arabic{section}}{1em}{}
\titleformat{\subsection}
{\normalfont\Large\bfseries}{}{1em}{}

\setlength{\droptitle}{-3em} 
\title{\textbf{ИДЗ №2}}
\author{Винницкая Дина Сергеевна}
\date{Группа: Б9122-02.03.01сцт}

\geometry{a4paper, margin=2cm}

\usepackage{setspace}
\onehalfspacing  

\renewcommand{\baselinestretch}{1.6} 

\begin{document}
	
	\maketitle

\subsection{Задание 5.16. (б)}
 Пусть $A: \ell_p \to \ell_p$, 
\[
Ax = \left\{ \sum_{j=1}^{\infty} a_{kj} \xi_j \right\}_{k=1}^{\infty}
\]
Тогда $A$ — сжимающее отображение в пространстве $\ell_p$, если
    \[
    \sup_{k \in \mathbb{N}} \sum_{j=1}^{\infty} |a_{kj}| < 1 \quad \text{при } p = \infty;
    \]

\subsection{Решение}

Пусть $x = (\xi_j) \in \ell_\infty$. Это означает, что существует константа $M$, такая что для всех $j \in \mathbb{N}$ выполнено $|\xi_j| \leq M$. Другими словами, последовательность $x$ ограничена по модулю, и её супремум конечен:
\[
\|x\|_\infty = \sup_j |\xi_j| < \infty
\]

Теперь рассмотрим, как действует оператор $A$ на такой вектор $x$. Его $k$-ая координата равна:
\[
(Ax)_k = \sum_{j=1}^\infty a_{kj} \xi_j
\]

Нас интересует, насколько может быть большим модуль этой суммы. Применим неравенство треугольника:
\[
|(Ax)_k| = \left| \sum_{j=1}^\infty a_{kj} \xi_j \right| \leq \sum_{j=1}^\infty |a_{kj}| \cdot |\xi_j|
\]

Так как $|\xi_j| \leq \|x\|_\infty$ для всех $j$, то:
\[
|(Ax)_k| \leq \left( \sum_{j=1}^\infty |a_{kj}| \right) \cdot \|x\|_\infty.
\]

Теперь найдём норму всего вектора $Ax$:
\[
\|Ax\|_\infty = \sup_k |(Ax)_k| \leq \sup_k \left( \sum_{j=1}^\infty |a_{kj}| \cdot \|x\|_\infty \right)
\]

Вынесем константу за скобки:
\[
\|Ax\|_\infty \leq \left( \sup_k \sum_{j=1}^\infty |a_{kj}| \right) \cdot \|x\|_\infty
\]

Обозначим:
\[
\alpha := \sup_k \sum_{j=1}^\infty |a_{kj}|
\]

Таким образом, мы получили:
\[
\|Ax\|_\infty \leq \alpha \cdot \|x\|_\infty.
\]

Теперь рассмотрим случай, когда $\alpha < 1$. В этом случае:
\[
\|Ax\|_\infty < \|x\|_\infty
\]
что и означает, что $A$ является \textit{сжимающим оператором} в пространстве $\ell_\infty$.

$$\qed$$


\end{document}



\end{document}
