\documentclass{article}

\usepackage[T2A]{fontenc}
\usepackage[utf8]{inputenc} 
\usepackage[english,russian]{babel} 
\usepackage{graphicx} 
\usepackage{amsmath}
\usepackage{amssymb}
\usepackage{cancel}
\usepackage{amsfonts}
\usepackage{amsthm}
\usepackage{titlesec}
\usepackage{titling} 
\usepackage{geometry}
\usepackage{pgfplots}
\usepackage{esint}
\pgfplotsset{compat=1.9}
\usepackage{amsthm}
\usepackage{array} 
\usepackage{longtable}
\usepackage{booktabs} 
\usepackage{xcolor}


\titleformat{\section}
{\normalfont\Huge\bfseries}{\arabic{section}}{1em}{}
\titleformat{\subsection}
{\normalfont\Large\bfseries}{}{1em}{}

\setlength{\droptitle}{-3em} 
\title{\vspace{-1cm}ИДЗ №2 \\ по дисциплине "Уравнения математической физики"}
\author{Винницкая Дина Сергеевна}
\date{Группа: Б9122-02.03.01сцт}

\geometry{a4paper, margin=2cm}

\usepackage{setspace}
\onehalfspacing  

\renewcommand{\baselinestretch}{1.6} 

\begin{document}
	
	\maketitle
\section*{Вариант №11}
\section*{Задание 1}

Решить краевую задачу для уравнения Лапласса $\Delta u = 0$ \\ \\
Вне круга $ r \geq a $, $ 0 \leq \varphi \leq 2\pi $, с граничным условием $ u(a, \varphi) = 14(\cos^3\varphi + \sin^2\varphi) $.
\section*{Решение}



Рассмотрим уравнение Лапласа вне круга радиуса $ a > 0 $:

$$
\Delta u = 0, \quad r \geq a, \quad \varphi \in [0, 2\pi],
$$

с граничным условием:

$$
u(a, \varphi) = 14(\cos^3\varphi + \sin^2\varphi).
$$


В полярных координатах $(r, \varphi)$ уравнение Лапласа имеет вид:

$$
\frac{1}{r} \frac{\partial}{\partial r}\left(r \frac{\partial u}{\partial r}\right) + \frac{1}{r^2} \frac{\partial^2 u}{\partial \varphi^2} = 0.
$$

Общее ограниченное при $r \to \infty$ решение уравнения Лапласа вне круга радиуса $a$:

$$
u(r, \varphi) = A_0 + \sum_{n=1}^\infty \left( A_n \left( \frac{a}{r} \right)^n \cos(n\varphi) + B_n \left( \frac{a}{r} \right)^n \sin(n\varphi) \right).
$$
Подставляем граничное условие:

$$
u(a, \varphi) = 14(\cos^3\varphi + \sin^2\varphi).
$$
Используем тригонометрические тождества:

- $\cos^3\varphi = \frac{3}{4} \cos\varphi + \frac{1}{4} \cos(3\varphi),$
- $\sin^2\varphi = \frac{1 - \cos(2\varphi)}{2}.$

Тогда:

$$
u(a, \varphi) = 14\left( \frac{3}{4} \cos\varphi + \frac{1}{4} \cos(3\varphi) + \frac{1 - \cos(2\varphi)}{2} \right)
= 7 + \frac{21}{2} \cos\varphi - 7 \cos(2\varphi) + \frac{7}{2} \cos(3\varphi).
$$
Сравниваем с рядом Фурье:

$$
u(a, \varphi) = A_0 + \sum_{n=1}^\infty \left( A_n \cos(n\varphi) + B_n \sin(n\varphi) \right).
$$
Получаем значения коэффициентов:

$$
A_0 = 7,\quad A_1 = \frac{21}{2},\quad A_2 = -7,\quad A_3 = \frac{7}{2},\quad A_n = 0\ (n \geq 4),\quad B_n = 0\ (\forall n).
$$
Подставляем найденные коэффициенты в общее выражение:
$$
u(r, \varphi) = 7 + \frac{21}{2} \cdot \frac{a}{r} \cos\varphi - 7 \cdot \left( \frac{a}{r} \right)^2 \cos(2\varphi) + \frac{7}{2} \cdot \left( \frac{a}{r} \right)^3 \cos(3\varphi)
$$

\subsection*{Ответ $\quad u(r, \varphi) = 7 + \frac{21}{2} \cdot \frac{a}{r} \cos\varphi - 7 \cdot \left( \frac{a}{r} \right)^2 \cos(2\varphi) + \frac{7}{2} \cdot \left( \frac{a}{r} \right)^3 \cos(3\varphi)$}

\end{document}




