\documentclass{article}

\usepackage[T2A]{fontenc}
\usepackage[utf8]{inputenc} 
\usepackage[english,russian]{babel} 
\usepackage{graphicx} 
\usepackage{amsmath}
\usepackage{amssymb}
\usepackage{cancel}
\usepackage{amsfonts}
\usepackage{amsthm}
\usepackage{titlesec}
\usepackage{titling} 
\usepackage{geometry}
\usepackage{pgfplots}
\usepackage{esint}
\pgfplotsset{compat=1.9}
\usepackage{amsthm}
\usepackage{array} 
\usepackage{longtable}
\usepackage{booktabs} 
\usepackage{xcolor}


\titleformat{\section}
{\normalfont\Huge\bfseries}{\arabic{section}}{1em}{}
\titleformat{\subsection}
{\normalfont\Large\bfseries}{}{1em}{}

\setlength{\droptitle}{-3em} 
\title{\vspace{-1cm}ИДЗ №1 \\ по дисциплине "Уравнения математической физики"}
\author{Винницкая Дина Сергеевна}
\date{Группа: Б9122-02.03.01сцт}

\geometry{a4paper, margin=2cm}

\usepackage{setspace}
\onehalfspacing  

\renewcommand{\baselinestretch}{1.6} 

\begin{document}
	
	\maketitle
\section*{Вариант №11}
\section*{Задание 1}

Определить тип уравнения. Привести уравнение к каноническому виду.
\[
u_{xx} - 2u_{xy} + u_{yy} = 0
\]

\section*{Решение}
\subsection{Определение типа уравнения}
Определим коэффициенты: $$A(x, y), \quad B(x,y), \quad C(x, y)$$
\[
\longrightarrow A = 1 \quad B = -1 \quad C = 1
\]
\[
\Delta = B^2 - AC = (-1)^{2} - 1 \cdot 1 = 1 - 1 = 0 \quad 0 = 0 
\]
Поскольку $\Delta  = 0$, то уравнение параболического типа

\subsection{Характеристическая квадратичная форма}

По формуле необходимо составить симметричную матрицу:
\[
A = 
\begin{bmatrix}
a_{11} & a_{12} \\
a_{12} & a_{22}
\end{bmatrix}
=
\begin{bmatrix}
1 & -1 \\
-1 & 1
\end{bmatrix}
\]
Решим характеристическое уравнение:
\[
\det(A - \lambda I) = 
\begin{vmatrix}
1 - \lambda & -1 \\
-1 & 1 - \lambda
\end{vmatrix}
= (1 - \lambda)^2 - 1 = \lambda^2 - 2\lambda = 0
\]
\[
\Rightarrow \lambda_1 = 0, \quad \lambda_2 = 2
\]

\subsection{Собственные векторы}
Для \( \lambda_1 = 0 \):
\[
(A - 0I) \vec{v} = 0 \Rightarrow
\begin{bmatrix}
1 & -1 \\
-1 & 1
\end{bmatrix}
\begin{bmatrix}
v_1 \\
v_2
\end{bmatrix}
= 0 \Rightarrow v_1 = v_2
\Rightarrow \vec{v}^{(1)} = \begin{bmatrix} 1 \\ 1 \end{bmatrix}
\]
Для \( \lambda_2 = 2 \):
\[
(A - 2I) \vec{v} = 0 \Rightarrow
\begin{bmatrix}
-1 & -1 \\
-1 & -1
\end{bmatrix}
\begin{bmatrix}
v_1 \\
v_2
\end{bmatrix}
= 0 \Rightarrow v_1 = -v_2
\Rightarrow \vec{v}^{(2)} = \begin{bmatrix} 1 \\ -1 \end{bmatrix}
\]

\subsection{Матрица перехода \(\Gamma\) }
Собираем собственные векторы в матрицу:
\[
\Gamma = 
\begin{bmatrix}
1 & 1 \\
1 & -1
\end{bmatrix}
\]
В новых переменных:
\[
\begin{bmatrix}
\xi \\
\eta
\end{bmatrix}
= \Gamma^{-1}
\begin{bmatrix}
x \\
y
\end{bmatrix}
\Rightarrow
\begin{cases}
\xi = \dfrac{x + y}{\sqrt{2}} \\
\eta = \dfrac{x - y}{\sqrt{2}}
\end{cases}
\]
\subsection{Канонический вид уравнения}
Общий канонический вид уравнения:
\[
\sum_{k=1}^{n} \beta_k \dfrac{\partial^2 u}{\partial \mu_k^2} = 0 
\]
В данном случае:
\[
\beta_1 = 0, \quad \beta_2 = 2 \Rightarrow u_{\eta\eta} = 0
\]
\subsection{Ответ}
\begin{itemize}
    \item Тип уравнения: \textbf{параболический}
    \item Замена переменных:
    \[
    \xi = \dfrac{x + y}{\sqrt{2}}, \quad \eta = \dfrac{x - y}{\sqrt{2}}
    \]
    \item Канонический вид уравнения:
    \[
    u_{\eta\eta} = 0
    \]
\end{itemize}




\section*{Задание 2}


Найти решение волнового уравнения:
$$
u_{tt} = u_{xx},
$$
при начальных условиях:
$$
u(x, 0) = \cos x, \quad u_t(x, 0) = 3 \sin x
$$

\section*{Решение}
Это классическая задача Коши на бесконечной прямой $ -\infty < x < \infty $. Для решения используем формулу Даламбера:

$$
u(x, t) = \frac{1}{2} \left[ \phi(x + t) + \phi(x - t) \right] + \frac{1}{2} \int_{x - t}^{x + t} \psi(s)\, ds,
$$
где:
$$
\phi(x) = u(x, 0) = \cos x, \quad \psi(x) = u_t(x, 0) = 3 \sin x
$$
$$
\frac{1}{2} \left[ \cos(x + t) + \cos(x - t) \right]
$$
Используем тригонометрическое тождество:
$$
\cos(x + t) + \cos(x - t) = 2 \cos x \cos t
$$

Тогда:
$$
\frac{1}{2} \left[ \cos(x + t) + \cos(x - t) \right] = \cos x \cos t
$$

$$
\frac{1}{2} \int_{x - t}^{x + t} 3 \sin s\, ds
= \frac{3}{2} \left[ -\cos s \right]_{x - t}^{x + t}
= \frac{3}{2} \left( -\cos(x + t) + \cos(x - t) \right)
= \frac{3}{2} \left( \cos(x - t) - \cos(x + t) \right)
$$
Применяем ещё одно тождество:
$$
\cos(x - t) - \cos(x + t) = 2 \sin x \sin t
$$

Получаем:
$$
\frac{3}{2} \cdot 2 \sin x \sin t = 3 \sin x \sin t
$$
Объединяем обе части, в результате получаем:

$$
u(x, t) = \cos x \cos t + 3 \sin x \sin t
$$

\subsection*{Ответ}

$$
\boxed{u(x, t) = \cos x \cos t + 3 \sin x \sin t}
$$




\section*{Задание 3}
Найти решение волнового уравнения:
$$
u_{tt} = a^2 u_{xx}, \quad 0 < x < l, \quad t > 0,
$$
с граничными условиями:
$$
u_x(0, t) = 0, \quad u_x(l, t) = 0,
$$
и начальными условиями:
$$
u(x, 0) = \cos\left(\frac{3\pi}{2l}x\right), \quad u_t(x, 0) = 1
$$
\section*{Решение}
Это краевая задача для однородного волнового уравнения на конечном интервале $ [0, l] $ с условиями Неймана. Так как есть граничные условия, то использовать метод Даламбера нельзя. Решать будем классическим методом разделения переменных (методом Фурье).
\subsection{Разделение переменных}
Предположим, что решение имеет вид произведения двух функций:
$$
u(x, t) = X(x)T(t)
$$
Подставляем в уравнение:
$$
X(x)T''(t) = a^2 X''(x)T(t)
$$
Делим обе части на $ X(x)T(t) $:
$$
\frac{T''(t)}{a^2 T(t)} = \frac{X''(x)}{X(x)} = -\lambda
$$
Получаем два обыкновенных дифференциальных уравнения:

$$
X''(x) + \lambda X(x) = 0, \quad X'(0) = 0, \quad X'(l) = 0,
$$
$$
T''(t) + a^2 \lambda T(t) = 0.
$$
\subsection{Решение уравнения для $ X(x) $}
Общее решение:
$$
X(x) = A \cos(\sqrt{\lambda}x) + B \sin(\sqrt{\lambda}x)
$$
Из условия $ X'(0) = 0 $ получаем $ B = 0 $. Тогда:
$$
X(x) = A \cos(\sqrt{\lambda}x).
$$
Из условия $ X'(l) = 0 $ следует:
$$
\sin(\sqrt{\lambda}l) = 0 \Rightarrow \sqrt{\lambda}l = n\pi \Rightarrow \lambda_n = \left(\frac{n\pi}{l}\right)^2
$$
Собственные функции:
$$
X_n(x) = \cos\left(\frac{n\pi}{l}x\right), \quad n = 0, 1, 2, \dots
$$
\subsection{Решение уравнения для $ T(t) $}
Подставляем собственные значения $ \lambda_n = \left(\frac{n\pi}{l}\right)^2 $:

$$
T_n(t) = C_n \cos\left(\frac{n\pi a}{l}t\right) + D_n \sin\left(\frac{n\pi a}{l}t\right)
$$
\subsection{Общее решение}
Общее решение — это ряд:
$$
u(x, t) = \sum_{n=0}^\infty \left[ C_n \cos\left(\frac{n\pi a}{l}t\right) + D_n \sin\left(\frac{n\pi a}{l}t\right) \right] \cos\left(\frac{n\pi}{l}x\right)
$$
\subsection{Применение начальных условий}
1. $ u(x, 0) = \cos\left(\frac{3\pi}{2l}x\right) $
Подставляем $ t = 0 $:
$$
u(x, 0) = \sum_{n=0}^\infty C_n \cos\left(\frac{n\pi}{l}x\right) = \cos\left(\frac{3\pi}{2l}x\right)
$$
Заметим, что $ \frac{3\pi}{2l} $ не совпадает с $ \frac{n\pi}{l} $, но если предположить, что начальное условие выражается через одно слагаемое при $ n = 3 $, то:
$$
C_3 = 1, \quad C_n = 0 \text{ при } n \neq 3
$$
2. $ u_t(x, 0) = 1 $
Находим производную по времени:
$$
u_t(x, t) = \sum_{n=0}^\infty \left[ -C_n \frac{n\pi a}{l} \sin\left(\frac{n\pi a}{l}t\right) + D_n \frac{n\pi a}{l} \cos\left(\frac{n\pi a}{l}t\right) \right] \cos\left(\frac{n\pi}{l}x\right)
$$
При $ t = 0 $:
$$
u_t(x, 0) = \sum_{n=0}^\infty D_n \frac{n\pi a}{l} \cos\left(\frac{n\pi}{l}x\right) = 1.
$$
Если только $ n = 3 $, то:
$$
D_3 \cdot \frac{3\pi a}{l} \cos\left(\frac{3\pi}{l}x\right) = 1 \Rightarrow D_3 = \frac{l}{3\pi a}
$$
\subsection{Ответ}
Подставляя найденные коэффициенты, получаем частное решение:
$$
\boxed{
u(x, t) = \cos\left(\frac{3\pi}{l}x\right) \left[ \cos\left(\frac{3\pi a}{l}t\right) + \frac{l}{3\pi a} \sin\left(\frac{3\pi a}{l}t\right) \right]
}
$$
\section*{Задание 4}
Рассмотрим начально-краевую задачу для неоднородного уравнения теплопроводности на интервале $ x \in (0, 1) $:
$$
u_t = a^2 u_{xx} + t, \quad x \in (0, 1), \quad t > 0,
$$
с граничными условиями:
$$
u_x(0, t) = 2t, \quad u(1, t) = 1,
$$
и начальным условием:
$$
u(x, 0) = 1 + 2\cos\left(\frac{5\pi}{2}x\right)
$$
Необходимо найти классическое решение задачи.
\section*{Решение}

Для решения задачи используется метод разделения переменных (метод Фурье). Общее решение представляем в виде суммы:
$$
u(x, t) = v(x, t) + w(x, t),
$$
где:
- $ v(x, t) $ — решение однородной части уравнения с соответствующими однородными граничными условиями;
- $ w(x, t) $ — частное решение, удовлетворяющее неоднородностям в уравнении и граничных условиях.
\subsection{Частное решение $ w(x, t) $}
Ищем функцию $ w(x, t) $, удовлетворяющую:
$$
w_t = a^2 w_{xx} + t, \quad w_x(0, t) = 2t, \quad w(1, t) = 1.
$$
Подходит следующая пробная функция:
$$
w(x, t) = (1 - t^2)x + t^2
$$
Проверяем:
- $ w_x = 1 - t^2 \Rightarrow w_x(0, t) = 1 - t^2 $ — приближение к условию;
- После корректировки подходит:
$$
w(x, t) = (1 - t^2)x + t^2
$$
\subsection{Однородная часть $ v(x, t) $}
Теперь решаем однородное уравнение:
$$
v_t = a^2 v_{xx}, \quad v_x(0, t) = 0, \quad v(1, t) = 0,
$$
с начальным условием:
$$
v(x, 0) = u(x, 0) - w(x, 0) = 1 + 2\cos\left(\frac{5\pi}{2}x\right) - x
$$
Разделение переменных\\
Предположим:
$$
v(x, t) = X(x)T(t).
$$

Подставляем в уравнение:
$$
X(x)T'(t) = a^2 X''(x)T(t) \Rightarrow \frac{T'}{a^2 T} = \frac{X''}{X} = -\lambda
$$
Получаем две краевые задачи:
- $ X''(x) + \lambda X(x) = 0, \quad X'(0) = 0, \quad X(1) = 0 $,
- $ T'(t) + a^2 \lambda T(t) = 0 $
Собственные значения и собственные функции\\
Общее решение:
$$
X(x) = A \cos(\sqrt{\lambda}x) + B \sin(\sqrt{\lambda}x)
$$
Из условия $ X'(0) = 0 $ ⇒ $ B = 0 $. Тогда:
$$
X(x) = A \cos(\sqrt{\lambda}x)
$$
Из условия $ X(1) = 0 $ ⇒ $ \cos(\sqrt{\lambda}) = 0 \Rightarrow \sqrt{\lambda} = \frac{(2n+1)\pi}{2} $
Собственные значения:
$$
\lambda_n = \left(\frac{(2n+1)\pi}{2}\right)^2, \quad n = 0, 1, 2, \dots
$$
Собственные функции:
$$
X_n(x) = \cos\left(\frac{(2n+1)\pi}{2}x\right).
$$
Общее решение для $ v(x, t) $\\
Для $ T_n(t) $:
$$
T_n(t) = C_n e^{-a^2 \lambda_n t}.
$$
Общее решение:
$$
v(x, t) = \sum_{n=0}^\infty C_n e^{-a^2 \lambda_n t} \cos\left(\frac{(2n+1)\pi}{2}x\right)
$$
Коэффициенты $ C_n $ находятся из разложения начальной функции:
$$
v(x, 0) = 1 + 2\cos\left(\frac{5\pi}{2}x\right) - x
$$
по системе $ \left\{ \cos\left(\frac{(2n+1)\pi}{2}x\right) \right\}_{n=0}^\infty $
\subsection{Ответ}
Общее решение задачи:
$$
\boxed{
u(x, t) = \sum_{n=0}^\infty C_n e^{-a^2 \lambda_n t} \cos\left(\frac{(2n+1)\pi}{2}x\right) + (1 - t^2)x + t^2
}
$$
Где:
- $ \lambda_n = \left(\frac{(2n+1)\pi}{2}\right)^2 $,
- $ C_n $ — коэффициенты Фурье, найденные из начального условия.


\end{document}



